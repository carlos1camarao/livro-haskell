\chapter{Funtores, Funtores Aplicativos e Mônadas}

%Funtores, funtores aplicativos e mônadas são abstrações representadas
%em Haskell como classes que permitem aplicação de funções em dados que
%estão encapsulados (ou seja, em dados que são instâncias de $a$ e
%estão armazenados em $f\: a$, em instância de algum construtor $f$):


     class Functor f where
          fmap:: (a->b) -> f a -> f b

     fmap g fa permite aplicar função (g) em um valor encapsulado fa
     (obtendo o resultado, depois da aplicação, encapsulado).

  2. class Functor f => Applicative f where
       pure  :: a -> f a
       (<*>) :: f (a -> b) -> f a -> f b

      (<*>) fg fa é similar a fmap: a diferença é que a função
      aplicada (fg) também está encapsulada. O efeito é, novamente,
      apenas aplicar a função ao valor encapsulado (obtendo o
      resultado encapsulado).

      pure apenas encapsula um valor (possivelmente, funcional).

   3. class Applicative f => Monad f where
        return :: a -> f a
        (>>=)  :: f a -> (a -> f b) -> f b

      return é o mesmo que pure.

      (>>=) fa g recebe valor encapsulado (fa) e passa o valor
      *desencapsulado* (a) à função g, mas exige que o resultado de g
      seja um valor encapsulado.

É importante notar que o uso de funções com efeitos colaterais por
meio dessas funções garante que o resultado final obtido esteja sempre
encapsulado; o encapsulamento (com o construtor IO, ou com o
construtor ST) também está encapsulado. É o encapsulamento e a
garantia de que os valores finais obtidos vão estar sempre
encapsulados que possibilita a manipulação de funções com efeito
colateral em uma linguagem pura.
