%\input{introd}

%Considere a definição e uso de tipos recursivos em linguagens
%imperativas, como \C\ e \CMM.  ...

%[[ Aqui vem exemplos, talvez começando com o tipo listas, ou árvores
%binárias, e talvez depois outros... ]]

%\input{tipos-basicos}

%\input{tipos-algebricos}

%\input{verificacao-e-inferencia-de-tipos}
\chapter{Tipos}

\section{A regra básica}
\label{Regra-basica}

A regra:

  \[ \frac{f : A \rightarrow B \:\:\: x : A}{f\: x : B} \]

é a regra fundamental usada por um compilador na verificação da
correção e na inferência dos tipos usados em programas.

Note que existe uma correspondência básica entre tipos e termos
lógicos (termos da lógica proposicional)
\cite{Martin-Lof:Constructive-math-and-programming,Sorensen-Urzyczyn:2006:Lectures-Curry-Howard-Iso}.
  Se consideramos tipos ($A$, $B$, $A \rightarrow B$) como proposições
  lógicas, e expressões ($f$, $x$, $f \:x$) como provas
  (demonstrações) dessas proposições, a regra acima é a regra
  fundamental (chamada de {\em modus ponens\/}) usada em lógica para
  dedução de novas formas válidas a partir de formas válidas
  existentes.

A correspondência entre proposições lógicas e tipos, e entre provas
(demonstrações) e expressões, é conhecida como ``correspondência de
Curry-Howard'' (também comumente chamada de ``isomorfismo de
Curry-Howard'')
\cite{Martin-Lof:Constructive-math-and-programming,Sorensen-Urzyczyn:2006:Lectures-Curry-Howard-Iso}.
%Foi ....

No caso de linguagens monomórficas, esta é a regra mais fundamental, e
as demais podem ser consideradas como auxiliares. Por exemplo, para
deduzir que \ina{(not True)} tem tipo \ina{Bool}, precisamos apenas
considerar ou saber que \ina{not} tem tipo \ina{(Bool->Bool)} e
\ina{True} tem tipo \ina{Bool}. Essa consideração ou conhecimento usa
tipos existentes em um contexto, nesse caso tipos pré-definidos de
constantes e funções pré-definidas.

% (Falar de duplas/tuplas e currying?) 

\section{Polimorfismo paramétrico}

No caso de linguagens polimórficas, essa regra também deve ser
aplicada, mas após {\em instanciação\/} do tipo do domínio da função
para que fique igual ao tipo do argumento. No caso da inferência de
tipos, essa instanciação é feita por uma função de {\em unificação\/},
que calcula a substituição mínima que se pode aplicar para que o tipo
do domínio da função e o tipo do argumento se tornem iguais.

Mais detalhadamente o processo de inferência de tipos usa a seguinte
regra de aplicação, que vamos chamar de regra (\APL): para que uma
função (potencialmente polimórfica) {\tt $f$: $t$} possa ser aplicada
a um argumento {\tt $x$: $t'$}, devemos {\em unificar\/} $t$ com
$t_A\rightarrow t_B$ (ou seja, $t$ deve ter um tipo funcional) e $t_A$
com $t'$ (ou seja, o tipo do argumento deve ser unificado com o tipo
do domínio da função); o tipo resultante é o tipo $t_B$ resultante da
unificação. Isso vai ser explicado mais precisamente na seção
seguinte.

O processo de inferência de tipos de um compilador atribui a valores
que ocorrem em expressões tipos que são conhecidos e, sempre que um
tipo de um valor ainda não é conhecido, como por exemplo o tipo de um
parâmetro formal de uma função, uma nova variável de tipo, ainda não
usada, é (inicialmente) atribuída a esse valor. Depois disso, a
inferência de tipos consiste em determinar os tipos resultantes das
unificações especificadas na regra (\APL), que vai ser abordada mais
detalhadamente na seção seguinte.

\subsection{Unificação e Substitituição}

Dado um conjunto de variáveis $V$, um conjunto de constantes $C$ e um
conjunto de nomes de funções $F$, o conjunto de termos $T$ pode ser
definido pelo seguinte tipo algébrico recursivo:

\prog{\data\ T = Con K | Var V | App T T}
onde supomos que $K$ é um conjunto de constantes e $V$ de variáveis de
tipo.

Um tipo como $t_1 \cdots\ t_n$ é visto, nessa representação, como
 $\App\ t_1 (\ldots\ \App\ t_{n-1}\: t_n)$.
 
Uma substituição é uma função finita $S: V_0 \rightarrow T$ de
variáveis em termos, onde $V_0$ é o conjunto finito de variáveis que
ocorrem em $T$. A função \id\ é a função identidade, isto é, a função
definida por $\id\: x = x$.

Dado um conjunto finito de equações $E = \{ t_i=t_i'|i=1,\ldots,n\}$,
onde $t_i, t_i'$ são termos (valores de $T$), uma substituição
$\sigma$ é um unificador dos termos de $E$ se
$\sigma(t_i)=\sigma(t_i)'$, para $i=1,\ldots,n$.

Além disso, uma solução $\sigma$ é mais geral que outra $\sigma'$ se
existe uma substituição $\sigma_1$ tal que
$\sigma=\sigma_1\circ\sigma'$. Intuitivamente, um unificador mais
geral de um conjunto de equações $E$ é o modo mais simples de tornar
todas as equações do conjunto $E$ iguais.

O seguinte algoritmo retorna \Just\ $m$, onde $m$ é mapeamento que
representa o unificador mais geral para uma lista qualquer de pares de
tipo (cada par representando uma equação), se um unificador existir,
caso contrário retorna \Nothing\ ({\tt (.)} é o operador de composição
de funções e \ina{ocorreEm} testa ocorrência de variável em tipo):

\begin{center}
\begin{tabular}{l}
\begin{alg}{.9\textwidth}{white}
unif [] = id 
unif ((t, V v): eqs)          = unif ((V v, t): eqs)  
unif ((V v, t): eqs)                                  
  | t == V v                  = unif eqs              
  | v `ocorreEm` t            = Nothing               
  | otherwise                 = s' . s                
  where                                               
    s    = insert v t empty                           
    eqs' = ap s eqs                                   
    s'   = unif eqs'                                  
unif ((K k1, K k2): eqs)                              
  | k1 == k2                  = unif eqs              
  | otherwise                 = Nothing               
unif (App t1 t2, App t1' t2') = s' . s                
  where s  = unif t1 t1'                              
        s' = unif (app s t2) (app s t2')              
unif _                        = Nothing               
\end{alg}
\end{tabular}
\end{center}

As regras usadas na inferência de tipos de um compilador de uma
linguagem com polimorfismo paramétrico são baseadas em um {\em
  contexto de tipos\/}, que associam tipos a variáveis (constantes
podem ser consideradas como variáveis, com tipos a ela associados em
um contexto global). Contextos de tipos mapeiam variáveis a tipos (é
comum na literatura o uso de conjuntos de pares representados na forma
{\em variável : tipo\/}). Vamos usar, como é comum, a meta-variável
$\Gamma$ para representar contextos de tipos.

Tipos de variáveis podem ser monomórficos, ou {\em simples\/} ou
polimórficos, ou {\em quantificados\/}. Um tipo simples é formado por
uma constante ou uma variável, possivelmente aplicada a um ou mais
tipos simples. Ou seja, é da forma:
\[ T \texttt{ ::= } K \mid V \mid T\: \bar{T} \]

Usamos $t$ como para denotar valores do tipo $T$, $a$ ou $b$ para
denotar valores de tipo $V$ e $k$ para denotar valores do tipo $K$,
podendo essas variáveis ter possivelmente subscritos e aspas
simples. Usamos $\bar{X}$ para denotar sequências, possivelmente
vazias, de valores $X$, ou conjuntos de tais valores. Por exemplo,
$\bar{T}$ representa uma sequência de tipos, e um tipo polimórfico
$\sigma$, do tipo de conjuntos polimórficos $\Sigma$, é da forma
$\forall \bar{a}.\:t$; ou seja:
\[ \Sigma \texttt{ ::= } \forall \bar{V}.\: T \]

A notação $\Gamma, x:t$ representa o contexto $\Gamma'$ tal que
$\Gamma'(x') = t$ se $x'=x$, senão $\Gamma(x')$ (ou seja, $\Gamma'$ é
um contexto igual a $\Gamma$ mas $\Gamma'(x) = t$).

A função $\tv$ retorna o conjunto de variaveis livres de um tipo. Em
particular, $\tv(\forall\bar{a}.\:t) = \tv(t) - \bar{a}$.

A relação $\gen(t,\sigma,\Gamma)$ é a relação de generalização de $t$
para $\sigma$ em $\Gamma$, definida de forma que $\sigma =
\forall\bar{a}.\:t$, onde $\bar{a} = \tv(t) - \tv(\Gamma)$.

Um tipo $S(\sigma)$, ou simplesmente $S\sigma$, onde $S$a é uma
substituição tal que $S(a) = a$ se $a \not= a_i$, $S(a_i) = b_iq$,
para $i$ de 1 a $n$, representa o tipo resultante de substituir cada
$a_i$ por $b_i$, para $i$ de 1 a $n$. O tipo $S\sigma$ pode ser
denotado também por
 $\texttt{\symbol{91}}\bar{a}\texttt{ := }\bar{b}\texttt{\symbol{93}}\sigma$.

O algoritmo de inferência de tipos de uma linguagem como o núcleo da
linguagem \ML\ pode ser descrito usando as regras mostradas na Figura
\ref{Inf-tipos}, onde $\Gamma \vdash e: (t,S)$ significa que $e$ tem
tipo $t$ no contexto de tipos $\Gamma$. A substituição $S$ é usada
para unificação de tipos de expressões usadas em $e$. Esse algoritmo
foi definido por Robin Milner e Luís Damas \cite{DamasMilner82}, sendo
chamado de algoritmo $W$.

A regra (\VAR) retorna o tipo da variável no contexto de tipos
(supõe-se que um programa é um termo fechado, isto é, sem variáveis
{\em livres\/}, no sentido de $\lambda$-cálculo).

A regra (\APL) é a regra básica (ver seção \ref{Regra-basica}) da
inferência de tipos: usa unificação para tornar o tipo domínio da
função igual ao tipo do argumento.

A regra (\ABS) cria uma nova variável ($a$, na Figura \ref{Inf-tipos})
para o tipo do parâmetro, usando a substituição ($S$) obtida na
inferência do tipo do corpo da $\lambda$-abstração para especialização
do tipo do parâmetro. O tipo da $\lambda$-abstração é o tipo funcional
$S(a)\rightarrow t$, sendo $t$ o tipo inferido para a expressão $e$,
corpo da $\lambda$-abstração.

A regra (\LET) generaliza o tipo inferido para $e$ para obter o tipo
de $e'$, que é retornado como o tipo da expressão {\tt \llett\ $x$ =
  $e$ \iinn\ $e'$}. Note que a regra (\LET) é que permite a introdução
de tipos polimórficos; o uso de variáveis definidas na regra (\LET)
pode ser usada em contextos que requerem tipos distintos, como por
exemplo em:
\[ \texttt{\llett\ $f$ = $\lambda x.\:x$ \iinn\ ($f\: \True, f\:1$)} \]
onde $\True$ e $1$ têm tipos distintos (\Bool\ e \Integer), e o tipo
de $f$ pode ser instanciado (no algoritmo $W$, via unificação) para
\Bool $\rightarrow$ \Bool\ e para \Integer $\rightarrow$ \Integer.

\begin{figure}
  \[ \begin{array}{lr}
        \frac{\Gamma(x) = \forall\bar{a}.\:t  \:\:\:\: {\bar{b} \textrm{variáveis novas}}}
            {\Gamma \vdash x:\: (\symbol{91}\bar{a} {\texttt{:=}}\,\bar{b}\symbol{93}\,t\,,\:\iid)}
            & \texttt{(\VAR)} \\[.2cm]
        \frac{\Gamma \vdash f: \:(t,S)  \:\:\:\: S(\Gamma) \vdash e:\: (t'\,,\:S') \:\:\:\: a \textrm{variável nova}}
             {\Gamma \vdash f \: e: \:(S(a)\,,\:S_u \circ S' \circ S )}
            & S_u = \unif(\symbol{91}(t,\: t' \rightarrow a)\symbol{93}) \:\:\:\: {\tt (\APL)} \\[.2cm]
        \frac{\Gamma,x:a \vdash e:\:(t\,,\:S) \:\:\:\: a \textrm{variável nova}}
             {\Gamma \vdash \lambda x \texttt{->}\:e:\: (S(a)\texttt{->}\:t\,,\:S)} 
            & \texttt{(\ABS)} \\[.2cm]
        \frac{\Gamma \vdash e:\: (t\,,\:S) \:\:\: \gen(t,\sigma,S(\Gamma)) \:\:\: S(\Gamma),x:\sigma \vdash e':\: (t',S')}
             {\Gamma \vdash \llett\ x {\texttt{ = }} e\: \iinn\ e':\: (t'\,,\:S' \circ S)}
            & \texttt{(\LET)}
     \end{array}
  \]
\caption{Algoritmo W}
\label{Inf-tipos}
\end{figure}

A regra (\LET) (e a linguagem núcleo de \ML\ do algortimo $W$ definido
na Figura \ref{Inf-tipos}) não considera recursão: $x$ não pode
ocorrer em $e$ na expressão {\tt \llett\ $x$ = $e$ \iinn\ $e'$}.

%Para permitir definições recursivas, há duas possibilidades: a
%primeira é permitir a chamada {\em recursão polimórfica\/},
%inicialmente estudada por Mycroft \cite{Mycroft84}, mas isso leva a
%indecidibilidade da inferência de tipos (isso foi provado por Kfoury
%et al. \cite{KTU94}). O leitor interessado pode consultar também
%\cite{LuCarWFLP2001}.

No entanto, o uso do algoritmo definido na Figura \ref{Inf-tipos} não
é necessário para determinar o tipo de funções, na prática, uma vez
que ele é baseado fundamentalmente no uso do tipo de variáveis que
estão no contexto de tipos, pela introdução de novas variáveis de tipo
para o tipo de variáveis introduzidas em $\lambda$-abstrações, e pelo
uso de unificação em aplicações de funções a seus argumentos. Não é
necessário seguir todos os passos do algoritmo. Podemos usar a
seguinte técnica, que vamos chamar de
técnica-informal-de-inferência-de-tipos. Considere, por exemplo, a
tarefa de inferir o tipo da expressão:

\[ \lambda f.\:\lambda x.\:f\:(f\:x) \]\

Sabemos que o tipo dessa expressão é da forma:
\[ t_f \rightarrow t_x \rightarrow t_r \]
onde: \begin{tabular}[t]{l}
        $t_f$ (é uma variável que) representa o tipo de $f$,  \\
        $t_x$ (é (uma variável que) representa o tipo de $x$ e \\
        $t_r$ (é uma variável que representa) o tipo do resultado, $f\:(f\: x)$.
\end{tabular}
\newline
Introduzimos agora as informações que pudermos obter com os usos das variáveis 
$f$ e $x$ na expressão $f\:(f\:x)$, devido a unificação, que são:
\begin{itemize}
\item $f$ tem tipo funcional (pois $x$ é aplicado a $f$), ou seja, é da forma $t_1\rightarrow t_2$;
\item o tipo de $x$ é o mesmo do domínio de $f$, ou seja, $t_1 = t_x$;
\item o tipo do resultado de $f$ ($t_2$) é o mesmo do argumento de $f$ (pois o resultado, $f\:x$ é aplicado a $f$).
\end{itemize}.

Note: cada ocorrência de parâmetro de função --- i.e.~cada variável em
uma $\lambda$-abstração --- tem um tipo monomórfico (essa é uma
característica fundamental do sistema de tipos de \ML\ e \Haskell).

Usando essas informações, obtemos que o tipo de:
\[ \lambda f.\:\lambda x.\:f\:(f\:x) \]
é igual a (chamando $t_x$ simplesmente de $a$):
\[ (a \rightarrow a) \rightarrow a \rightarrow a \]

Essa técnica pode ser estendida para os casos de definições
recursivas, que envolvam mais de uma equação, (lembrando que
parâmetros têm tipos monomórficos e) considerando que:

\begin{itemize}
  \item uma equação {\tt $f\: p_1\: \ldots\: p_n$ = $e$} pode ser vista
    como tendo o tipo de uma $\lambda$-abstração: {\tt $\lambda p_1.\:
      \ldots\: \lambda p_n.\: e$}, sendo esse o tipo de $f$.
  \item toda definição:
    {\texttt{ \[\begin{array}[t]{ll}
                  f\: p_{11}\: \ldots\: p_{1n}\: & = \:e_1\\
                  \ldots & \\
                  f\: p_{m1}\: \ldots\: p_{mn}\: & = \:e_m
        \end{array}\]}}
      deve ser tal que os tipos de cada resultado $e_i$, assim como os
      tipos de cada parâmetro $p_{ij}$, para cada $i,j$
      ($i=1,\ldots,m$ e $j=1,\ldots,n$) são iguais: para isso os tipos
      de cada equação devem ser unificados (se não for possível
      unificar os tipos de cada equação, a definição não é bem
      tipada).
      
\end{itemize}

Por exemplo, considere a definição de \map:

\begin{center}
\begin{tabular}{l}
\begin{alg}{.9\textwidth}{white}
  map f []    = [] 
  map f (a:x) = f a : map f x
\end{alg}
\end{tabular}
\end{center}

Vamos calcular o tipo de \map\ inicialmente na primeira
equação. Poderíamos começar com $t_f \rightarrow t_l \rightarrow t_r$, sendo
$t_f$, onde:
\begin{quotation}
  \begin{tabular}{l}
        $t_f$ é (uma variável que representa) o tipo de $f$,  \\
        $t_x$ é (uma variável que representa) o tipo de {\tt []} e \\
        $t_r$ é (uma variável que representa) o tipo do resultado, também {\tt []}.
  \end{tabular}
 \end{quotation}
No entanto, como sabemos, pelo tipo de {\tt []} no contexto de tipos,
que {\tt []} tem tipo {\tt [$a$]}, sendo $a$ uma variável nova,
podemos considerar que o tipo de \map\ na primeira equação é:

{\ttfamily
  \[ t_f \rightarrow [a] \rightarrow [b] \]
}

Note que as ocorrências de {\tt []} têm tipos distintos, com variáveis
de tipo $a$ e $b$ novas, pois o tipo de {\tt []}, no contexto de
tipos, é polimórfico.

Na segunda equação, podemos atribuir a \ina{map} inicialmente o tipo:
$t_f \rightarrow t_l \rightarrow t_r$, sendo:
     $t_f$ variável que representa o tipo de $f$,  
     $t_l$ variável que representa o tipo de \ina{a:x} e
     $t_r$ variável que representa o tipo de \ina{f a:map f x}.
Temos:

\begin{itemize}
  \item sendo $t_l$ igual a {\tt [$a$]} (de acordo com o tipo de {\tt
    []} usado para esse parâmetro na primeira equação), o tipo de
    \ina{a} é igual a $a$ e o tipo de \ina{x} é igual a {\tt [$a$]}
    (pois o tipo de \ina{(:)} no contexto de tipos é igual a
    $a'\rightarrow \texttt{[$a'$]} \rightarrow \texttt{[$a'$]}$, para
    alguma variável nova $a'$, mas $a'$ deve ser igual a $a$, pois os
    tipos dos parâmetros nas duas equações devem ser iguais.

  \item $t_f$ deve ser um tipo funcional, pois \ina{f} é aplicado a
    \ina{a}, e o tipo do resultado de \ina{f} deve ser igual a $b$,
    igual ao tipo do elemento da lista {\tt [$b$]}: {\tt [$b$]} é o
    resultado de \ina{map f []} na primeira e na segunda equações, e
    igual ao tipo do resultado de \ina{map f x}.

\end{itemize}

Portanto, o tipo de \ina{map}, definido pelas duas equações acima, é:

\[ (a \rightarrow b) \rightarrow \texttt{[$a$]} \rightarrow \texttt{[$b$]}
\]

\subsection{Sobrecarga e Tipos restritos}
\label{Tipos-restritos}

Um valor sobrecarregado tem um tipo polimórfico restrito (ou
constrito) --- por exemplo, {\tt (==)} tem tipo
  \ina{Eq a => a -> a -> Bool}.

{\em Quando a sobrecarga é resolvida\/} (e apenas neste caso), devido
a instanciação do tipo polimórfico (via unificação), é preciso
verificar se há instâncias no contexto que satisfazem ao tipo do valor
sobrecarregado que foi instanciado:

\begin{itemize}
  \item se há apenas uma instância, a restrição é removida do tipo;
  \item se não ha nennhuma instância, ocorre um erro de tipo:
    insatisfazibilidade.
  \item se há duas ou mais instâncias, ocorre um erro de tipo: ambiguidade.
\end{itemize}

Por exemplo, a expressão {\tt 1 + '1'} usualmente gera um erro de
tipo, porque usualmente não há instância de \Num\ para o tipo \Char.

Em Haskell, infelizmente, não há, ainda, definição de quando uma
sobrecarga é resolvida, e esta condição (quando uma sobrecarga é
resolvida) é confundida com ambiguidade: em Haskell um tipo é ambíguo
se existe uma variável de tipo que ocorre em uma restrição e não
ocorre no tipo simples. Por exemplo, o tipo da expressão
\ina{(show . read)},
que é:
\begin{center}
\begin{tabular}{l}
\begin{alg}{.9\textwidth}{white}
  (Read a, Show a) => String -> String
\end{alg}
\end{tabular}
\end{center}
é considerado ambíguo, pois a variável de tipo \ina{a} ocorre em uma
restrição e não ocorre no tipo (simples) \ina{String -> String}.

Em Haskell, não há, assim, diferença entre insatisfazibilidade e
ambiguidade, nem entre sobrecarga resolvida e ambiguidade.

O teste de satisfazibilidade, que deve verificar se existe ou não uma
única instância que satisfaz a uma restrição, é em geral um problema
indecidível \cite{Smith91,VolpanoSmith91}. Existem várias opções de
compilação no GHC para restringir tipos polmórficos restritos
\cite{GHC}. No entanto, essas opcões poderiam ser evitadas por um
mecanismo baseado em uma medida do tamanho dos tipos que formam as
restrições, medida essa calculada durante o processo de testar
satisfazibilidade 
\cite{Ambig-and-cxt-dep-overloading-2013,Ambig-and-constrained-poly-2016}.

Quando há mais de uma equação em uma definição, as restrições de cada
equação devem ser consideradas (isto é, devem ser unidas) para gerar a
restrição final resultante da unificação de cada uma das
equações. Veja por exemplo o exercício resolvido \ref{Ex:merge}.

\section{Exercícios Resolvidos}

\begin{enumerate}

% \item Determine o tipo de $\lambda f.\:\lambda x.\:f\:(f\: x)$,
%   usando a técnica-informal-de-inferência-de-tipos.
%
%   {\em Solução\/}: Podemos ver que $f$ é uma função, digamos $a
%   \rightarrow b$ (pois $f$ é aplicada a $x$ e a $(f\:x)$), onde $a$ e
%   $b$ são variáveis novas. Podemos ver que $a$ é igual ao tipo de
%   $x$, e que o tipo de retorno de $f$ (igual a $b$) também é igual a
%   $a$, pois $f\:x$ (retorno de $f$) é também argumento de $f$.
%   Portanto, o tipo de $\lambda f.\:\lambda x.\:f\:(f\: x)$ é:
%
%     \[ (a \rightarrow a) \rightarrow a \rightarrow a \]

 \item \label{Ex:merge} Considere a definição de \ina{merge} a seguir:

\begin{center}
\begin{tabular}{l}
\begin{alg}{.9\textwidth}{white}
merge :: ([a], [a]) -> [a]
merge ([], y) = y                   -- (1)
merge (x, []) = x                   -- (2) 
merge (a:x, b:y)                    -- (3) 
  | a <= b    = a:merge (x, b:y)    -- (4)
  | otherwise = b:merge (a:x, y)    -- (5)
\end{alg}
\end{tabular}
\end{center}

Use a técnica-informal-de-inferência-de-tipos para determinar o tipo
de \ina{merge}.

{\it Solução\/}: O tipo de \ina{merge} é, inicialmente, igual a $t_p
\rightarrow t_r$, onde $t_p$ é o tipo do argumento e $t_r$ o tipo do
resultado.  Vamos determinar o tipo de cada equação, depois vamos
considerar que o tipo de cada equação é o mesmo, e para isso vamos
usar unificação, conservando (isto é, unindo) as restrições existentes
nos tipos de cada equação.

O tipo de \ina{merge} na linha \ina{1} (primeira equação) é
  {\tt ([$a$],$t_y$) -> $t_y$}.

O tipo de \ina{merge} na linha \ina{2} (segunda equação) é {\tt
  ($t_x$,[$b$]) -> $t_x$}.

Na linha \ina{3} (são duas equações neste caso, uma para cada guarda),
o tipo de \ina{merge} é {\tt \Ord\ $a$ => ([$a$],[$a$]) -> [$a$]};
  note que:
  \begin{itemize}
    \item a restrição {\tt \Ord\ $a$} é inserida devido ao uso do
      operador sobrecarregado {\tt <=} (na linha \ina{4});
    \item inicialmente o tipo de \ina{merge} na linha \ina{3} seria
      {\tt ([$a$],[$b$]) -> $t_r$}, mas, tanto devido à comparação
      \ina{a <= b} --- o tipo de {\tt <=} é
      \ina{Ord a => a -> a -> Bool},
      portanto o tipo de \ina{a} e de \ina{b} em \ina{a <= b} têm que
      ser iguais --- quanto devido ao fato de que os tipos de $t_r$
      nas linhas \ina{4} e \ina{5} têm que ser iguais ($t_r$ é igual
      {\tt [$a$]} na linha \ina{4} e igual a {\tt [$b$]} na linha
      \ina{5}, o tipo de \ina{merge} na última ``equação'' é
          {\tt \Ord\ $a$ => ([$a$],[$a$]) -> [$a$])}.
  \end{itemize}

  Unificando os tipos das equações, obtemos o tipo de \ina{merge}:

    \[ \texttt{\Ord\ $a$ => ([$a$],[$a$]) -> [$a$]} \]

\end{enumerate}

\section{Exercícios}

\begin{enumerate}

\item Determine o tipo de $\lambda x.\:\lambda y.\:\lambda z.\:x\: z
  (y\: z)$, usando a técnica-informal-de-inferência-de-tipos.

\item Use a técnica-informal-de-inferência-de-tipos para determinar o
  tipo da função \ina{either} definida abaixo:

  \begin{enumerate}

  \item Sendo o tipo \Either\ é definido como:
\begin{center}
\begin{tabular}{l}
\begin{alg}{.9\textwidth}{white}
  data Either a b = Left a | Right b
\end{alg}
\end{tabular}
\end{center}  
defina como é inferido o tipo de \ina{either}, usando a
técnica-informal-de-inferência-de-tipos, definido como:
\begin{center}
\begin{tabular}{l}
\begin{alg}{.9\textwidth}{white}
    either f g (Left x)  =  f x
    either f g (Right y) =  g y
\end{alg}
\end{tabular}
\end{center}  

  \end{enumerate}
  
\item Defina expressão ou função com tipo:

  \begin{enumerate}

  \item {\tt (\Ord\ a, \Show\ a) -> \String}

  \item {\tt ($a$ -> $b$ -> $c$) -> $b$ -> $a$ -> $c$}

  \item {\tt ($a$ -> $b$, $a$ -> $c$) -> $a$ -> ($b$,$c$)}

  \item {\tt ($a$ -> $b$, $b$ -> $d$) -> ($a$,$b$) -> ($c$,$d$)}

  \item {\tt ($b$ -> $c$) -> ($a$->$b$) -> $a$ -> $c$}
  
  \item {\tt [($a$, $b$)] -> ([$a$], [$b$])}

  \item {\tt ($a$ -> $b$ -> $a$) -> $a$ -> [$b$] -> [$a$]}

  \end{enumerate}
    
\end{enumerate}
