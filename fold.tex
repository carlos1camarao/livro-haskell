\chapter{Funções de ordem superior}

Funções de ordem superior como \map\ e \filter\ permitem a
implementação de operações comuns em estruturas de dados: aplicar uma
função a todos os elementos da estrutura de dados e filtrar os
elementos que satisfazem a um predicado.

No entanto, estes são casos especiais de uma operação mais expressiva,
chamada de {\em fold} (ou {\em reduce\/}), que permite obter um
resultado pela aplicação de uma função binária aos elementos de uma
estrutura de dados para obtenção de um resultado final, a partir de um
valor inicial.

Vamos trabalhar nestas notas de aula com {\em fold\/}s sobre listas,
para as quais podemos ter \ina{foldr} (``fold right'', i.e. \ina{fold}
sobre os elementos da lista da direita para a esquerda, em outras
palavras, do último até o primeiro elemento da lista) e \ina{foldl}
(``fold left'', i.e. \ina{fold} dos elementos da lista da esquerda
para a direita, em outras palavras, do primeiro até o último
elemento).

\section{\foldr}

Consideremos primeiro \ina{foldr}. O tipo dessa função é:

\begin{center}
\begin{tabular}{l}
\begin{alg}{.5\textwidth}{white}
  (a -> b -> b) -> b -> [a] -> b}
\end{alg}
\end{tabular}
\end{center}
    
\ina{foldr f z} aplicado a uma lista {\tt [$x_1$, $\ldots$, $x_n$]}
fornece o resultado:

\begin{equation}
  x_1\: \symbol{96}f\,\symbol{96}\: (\ldots\: (x_{n-1}\: \symbol{96}f\,\symbol{96}\: (x_n\: \symbol{96}f\,\symbol{96}\: z))\ldots)
  \label{comportamento-foldr}
\end{equation}

Em outras palavras, o resultado de \ina{foldr f z} aplicado à lista
{\tt $x_1\:$:$\:$($\ldots$ ($x_n\:$:$\:$[]))} é obtido substituindo
{\tt []} por \ina{z} e {\tt (:)} por {\tt
  \symbol{96}\ina{f}\symbol{96}}.

O valor de tipo \ina{b} obtido em aplicações sucessivas de \ina{f} em
um {\em fold} é chamado de {\em acumulador\/}.

O resultado é calculado da direita para a esquerda, isto é, a primeira
aplicação é {\tt \ina{f} $x_n$ \ina{z}}, depois {\tt \ina{f} $x_{n-1}$
  $r_1$}, sendo $r_1$ o resultado obtido por essa primeira aplicação,
até {\tt \ina{f} $x_1$ $r_{n-1}$}, sendo esse o último e $r_{n-1}$ o
penúltimo resultado obtido com aplicações de $f$.

\subsection*{Exemplos}

Exemplos básicos de uso de \foldr\ são mostrados a seguir. Procure
exercitar, definindo você mesmo, antes de ver a definição.

Para isso, pense o seguinte: \foldr\ $f$ $z$ obtém um valor final
aplicando $f$ a partir do valor inicial $z$. É preciso definir $f$ e
$z$. Leve em conta que: $f$ recebe um elemento da lista, o resultado
(de tipo $r$) de fazer \foldr\ no restante da lista, e retorna um
valor (do mesmo tipo de $r$); $z$ é o valor retornado quando a lista é
vazia.

É útil também ter em mente como \foldr\ computa seu resultado, como
mostrado em (\ref{comportamento-foldr}).

\begin{enumerate}
\item soma de todos os elementos:

\begin{center}
\begin{tabular}{l}
\begin{alg}{.3\textwidth}{white}
  sum = foldr (+) 0
\end{alg}
\end{tabular}
\end{center}

\item multiplicação de todos os elementos:

\begin{center}
\begin{tabular}{l}
\begin{alg}{.3\textwidth}{white}
  prod = foldr (*) 1
\end{alg}
\end{tabular}
\end{center}

\item concatenação de todos os elementos (de uma lista de listas):

\begin{center}
\begin{tabular}{l}
\begin{alg}{.37\textwidth}{white}
  concat = foldr (++) []
\end{alg}
\end{tabular}
\end{center}

\item conjunção de todos os elementos: 

\begin{center}
\begin{tabular}{l}
\begin{alg}{.35\textwidth}{white}
  and = foldr (&&) True
\end{alg}
\end{tabular}
\end{center}

\item disjunção de todos os elementos: 

\begin{center}
\begin{tabular}{l}
\begin{alg}{.35\textwidth}{white}
  or = foldr (||) False
\end{alg}
\end{tabular}
\end{center}

\item máximo dos elementos (de uma lista não vazia):

\begin{center}
\begin{tabular}{l}
\begin{alg}{.5\textwidth}{white}
  maximum (a:x) = foldr max a x
\end{alg}
\end{tabular}
\end{center}

\item mínimo dos elementos (de uma lista não vazia):

\begin{center}
\begin{tabular}{l}
\begin{alg}{.5\textwidth}{white}
  minimum (a:x) = foldr min a x
\end{alg}
\end{tabular}
\end{center}

\item comprimento (número de elementos):

\begin{center}
\begin{tabular}{l}
\begin{alg}{.45\textwidth}{white}
  len = foldr (\ _ -> (+1)) 0
\end{alg}
\end{tabular}
\end{center}

\item \map:

\begin{center}
\begin{tabular}{l}
\begin{alg}{.4\textwidth}{white}
  map f = foldr (:) []
\end{alg}
\end{tabular}
\end{center}
  
\item \filter:

\begin{center}
\begin{tabular}{l}
\begin{alg}{.8\textwidth}{white}
  filter p = foldr (\ a -> if p a then (a:) else id) []
\end{alg}
\end{tabular}
\end{center}

\end{enumerate}

Mais exemplos são mostrados a seguir.

\begin{enumerate}

\item teste de pertinência a lista:

\begin{center}
\begin{tabular}{l}
\begin{alg}{.2\textwidth}{white}
    elem a = foldr ((||) . (==a)) False
\end{alg}
\end{tabular}
\end{center}
  
\item ordenação por inserção: 

\begin{center}
\begin{tabular}{l}
\begin{alg}{.5\textwidth}{white}
insertionSort = foldr ins []
  ins a []      = [a]
  ins a (b:x)
    | a <= b    = a:b:x
    | otherwise = b: ins a x
\end{alg}
\end{tabular}
\end{center}

\item inserção de valor (\ina{k}) entre dois elementos consecutivos de
  uma lista; na definição de \ina{intersperse} mostrada abaixo, um
  valor booleano é usado para não inserir \ina{k} após o último
  elemento:

\begin{center}
\begin{tabular}{l}
\begin{alg}{.8\textwidth}{white}
intersperse k = fst . foldr consIfTrue ([],(k,False))
  where consIfTrue a (x,(k,False)) = (a:x  ,(k,True))
        consIfTrue a (x,(k,b    )) = (a:k:x,(k,b   ))
\end{alg}
\end{tabular}
\end{center}

\end{enumerate}

\section{\foldl}

\ina{foldl f z} aplicado a uma lista {\tt [$x_1$, $\ldots$, $x_n$]} fornece o
resultado:
  \[ \texttt{(\ldots ((z \symbol{96}\ina{f}\symbol{96} $x_1$) \symbol{96}\ina{f}\symbol{96} $x_2$) \ldots \symbol{96}\ina{f}\symbol{96} $x_n$)}\]

O tipo de \ina{foldl} é:

\begin{center}
\begin{tabular}{l}
\begin{alg}{.5\textwidth}{white}
  (b -> a -> b) -> b -> [a] -> b
\end{alg}
\end{tabular}
\end{center}

A função \ina{f} em \ina{foldl f}, ao contrário de \ina{foldr}, recebe
o ``acumulador'' primeiro, e depois o elemento da lista.

\fold\ é útil em casos (não muito comuns) em que o primeiro argumento
é não estrito no segundo argumento, ou quando se deseja inverter a
ordem dos elementos da lista.

Exemplos de uso de \foldl:

\begin{enumerate}

\item inversão da ordem dos elementos da lista:

\begin{center}
\begin{tabular}{l}
\begin{alg}{.2\textwidth}{white}
    reverse = foldl (flip (:)) []
\end{alg}
\end{tabular}
\end{center}

\item subtração de todos elementos de lista a valor:

\begin{center}
\begin{tabular}{l}
\begin{alg}{.2\textwidth}{white}
  subtraiTodosDe n = foldl (-) n
\end{alg}
\end{tabular}
\end{center}

\end{enumerate}

\section{Ineficiência da complexidade de espaço da avaliação preguiçosa de \foldl}

Infelizmente, ao contrário da complexidade de tempo, a complexidade de
espaço da estratégia de avaliação preguiçosa não é ótima. Por exemplo,
a quantidade de espaço usada para obter o resultado de:

\[ \texttt{\foldl\ $f$ $z$ $x$} \]
é $O(n)$, onde $n$ é o tamanho da lista, quando $f$ é estrita.  A
complexidade de espaço usada para obter o resultado de
$\texttt{\foldl\ $f$ $z$ $x$}$ é também $O(n)$ quando $f$ é
estrita. Se a lista for grande, isso poderá acarretar problema de
falta de espaço suficiente para obter o resultado.

No entanto, no caso de \foldl, a complexidade de espaço usada pela
estratégia de avaliação estrita é $O(1)$. Isso ocorre, no cálculo de
$\texttt{\foldl\ $f$ $z$ $x$}$, sendo {\tt $x$ $=$
  [$x_1$,$\ldots$,$x_n$]}, porque o resultado $r_i$ de $f\:z\:x_i$ é
calculado imediatamente, em vez de ser salvo para ser calculado
posteriormente, no cálculo de $f\: r_1\: x_{i+1}$, para
$i=1,\ldots,n$, e isso evita que espaço de memória tenha que ser
alocado a cada $i$ de $1$ até $n$.

No caso de \foldr, a complexidade de espaço da avaliação preguiçosa
não pode ser melhorada (de $O(n)$ para uma complexidade menor) porque
a construção dos resultados tem que ser realizada a partir o primeiro
elemento da lista e terminar com o último elemento da lista (e isso
envolve um custo $O(n)$), de modo a permitir que os resultados sejam
calculados do último elemento da lista até o primeiro.

\section{\foldr\ versus \foldl}
  
Quando analisamos a eficiência, em termos de tempo e espaço, de
\foldr\ e \foldl, por exemplo para decidir quando usar uma ou outra
função, podemos chegar a conclusões interessantes.

Para isso, consideremos \foldr\ primeiro:

{\ttfamily
\[ \begin{array}{lllllllll}
      \foldr\ f\: z\:\: ( &\!\!\!\! x_1 & :   & (\ldots & :   & (x_n & :   & \texttt{[\,]}) & \ldots)) = \\ 
                          &\!\!\!\! x_1 & `f` & (\ldots & `f` & (x_n & `f` & z)  & \ldots)
\end{array}
\]}

Note:

\begin{enumerate}
    \item Se $f\:e$ não for estrita (isto é, se o resultado da
      avaliação de $e_1$ for suficiente para estabelecer o resultado
      de {\tt $f\:e_1\:e_2$}, ou seja, se a avaliação de $e_2$ puder
      não ser necessária para estabelecer o resultado de {\tt $e_1$
        `$f$` $e_2$}), a complexidade de tempo do cálculo de {\tt
        (\ina{foldr} $f$ $z$ $x$)} é sub-linear, pois o número de
      valores de $x$, a partir de $x_1$, necessários para que $f$
      estabeleça um resultado, será menor ou igual a $n$.

      Por exemplo, \ina{foldr (&&) True [False..]}  é igual a
      \ina{False} (apesar de \ina{[False..]}  ser infinita), e o tempo
      necessário para avaliação é constante (esse é o tempo necessário
      para o cálculo de \ina{False && l}, podendo \ina{l} ser qualquer
      lista, finita ou infinita).

      O mesmo acontece para a complexidade de espaço.

    \item Se, no entanto, se $f$ e $f\:e$ forem estritas (isto é, se o
      resultado da avaliação de $e_1$ e de $e_2$ forem necessários
      para a avaliação de {\tt $e_1$ `$f$` $e_2$}), a complexidade de
      tempo do cálculo de \ina{foldr} $f$ $z$ $x$ (onde $x$ é igual a
      \texttt{($x_1$ : ($\ldots$ : ($x_n$ : []) $\ldots$))}) é $O(n)$,
      onde $n$ é o tamanho de $x$.

      O mesmo acontece para a complexidade de espaço.

    \item No caso de \foldl, a função $f\:z e$ deve ser não-estrita no
      segundo argumento ($e$): i.e.~o resultado de $f\:z\:e$ deve
      poder ser obtido sem necessidade de avalizar $z$). Lembre-se:

      \[ \foldl\ f\: z\: ( x_1 : (\ldots : (x_n :  \texttt{[\,]}) \ldots)) = 
       \: f ( \ldots (f ( f\: z\: x_1 )\: x_2)\: \ldots\: x_n) \]

      Por exemplo, a avaliação de \ina{foldl (&&) True [False..]}  não
      termina, porque \ina{(&&)} é não-estrita no primeiro argumento
      (faz casamento de padrão no primeiro argumento) e não no
      segundo. Ou seja, é preciso o resultado de cada conjunção
      precisa ser obtido para obtençao da próxima conjunção (o que
      requer o processamento de uma lista infinita de valores iguais a
      \False.
      
\end{enumerate}

\section{Exercícios}


\begin{enumerate}

\item Escreva, usando \foldl\ ou \foldr\, uma função que recebe uma
  lista de cadeias de caracteres (valores do tipo \String) e retorna
  uma cadeia de caracteres que contém os 3 primeiros caracteres de
  cada cadeia.

  Por exemplo, ao receber {\tt ["Abcde", "1Abcde", "12Abcde", "123Abcde"]}
  deve retornar {\tt \symbol{34}Abc1Ab12A123\symbol{34}}.

\item Escreva, usando \foldr\ ou \foldl, uma função que recebe uma
  lista de pessoas (valores de tipo \Pessoa, veja definição a seguir)
  e retorna a soma das idades (valores do campo \idade) de todos os
  elementos da lista.

\begin{center}
\begin{tabular}{l}
\begin{alg}{.5\textwidth}{white}
   data Pessoa = Pessoa {nome::Nome, idade::Idade, id::RG}
   type Nome   = String
   type Idade  = Integer
   type RG     = String
\end{alg}
\end{tabular}
\end{center}

\item Escreva, usando \foldr\ ou \foldl, uma função que recebe uma
  lista de valores de tipo Item, da definição acima, e retorna o nome
  da pessoa mais nova da lista.

\item Escreva, usando \foldl\ ou \foldr, uma função que recebe uma
  lista de cadeias de caracteres (valores do tipo \String) e retorna
  uma cadeia de caracteres que contém os 3 primeiros caracteres de
  cada cadeia removidos se não forem letras, ou com as letras em caixa
  alta se forem letras, e com os demais caracteres depois dos 3
  primeiros sem alteração.

  Por exemplo, ao receber {\tt ["Abcde", "1Abcde", "12Abcde", "123Abcde"]}
  deve retornar {\tt \symbol{34}ABCdeABcdeAbcdeAbcde\symbol{34}}.

\item Explique porque \foldr\ $f$ $x$ pode não percorrer toda a lista
  $x$, ao passo que toda a lista x é sempre percorrida, no caso de
  foldl.

\item A função \remdups\ remove elementos iguais adjacentes de uma
  lista, conservando só um dos elementos.

   Por exemplo, {\tt \remdups\ [1,2,2,3,3,3,1,1] = [1,2,3,1]}.

   Defina \remdups\ usando \foldr\ ou \foldl. 

\end{enumerate}

%\end{document}
