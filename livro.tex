\documentclass[a4paper,10pt]{book}

\usepackage{fancyheadings}

%\usepackage{amsfonts}
%\usepackage{amsmath}
%\usepackage{amssymb}
%\usepackage{amsthm}

%\usepackage{proof}

%\usepackage{psfig}
\usepackage{fancybox}
\usepackage{makeidx}

\usepackage{graphicx}                               % graphics

\usepackage{indentfirst}  % indentation at paragraph beggining
%\usepackage{hyperref}     % for links in pdf and webpage
\usepackage{makeidx}

%-- hevea and page styling -------------------------
\usepackage{hevea}
\usepackage{hevea-epsfbox-def}  

\usepackage{fancybox}
\usepackage{fancyheadings}
\usepackage{fancysection}            % hevea styling for sections

\usepackage[utf8]{inputenc}
\usepackage[portuguese]{babel}

\usepackage[usenames,dvipsnames]{xcolor}


%--- page style ----------------------------------
\setlength{\headheight}{15.2pt}
\pagestyle{fancyplain}
\addtolength\headwidth{\marginparsep}
\renewcommand{\chaptermark}[1]
             {\markboth{#1}{}}
\renewcommand{\sectionmark}[1]
             {\markright{\thesection\ #1}}
\lhead[\fancyplain{}{\sfseries\thepage}]
     {\fancyplain{}{\sfseries\rightmark}}
\rhead[\fancyplain{}{\sfseries\leftmark}]
      {\fancyplain{}{\sfseries\thepage}}
\cfoot{}

%----- text format parameters -----
\setlength\textwidth{150mm}
\setlength\textheight{241mm}
\setlength\oddsidemargin{5.4mm}
\setlength\evensidemargin{5.4mm}
\setlength\marginparsep{5mm}
\setlength\marginparwidth{15mm}
\addtolength\voffset{-5mm}
\addtolength\topmargin{-3mm}
\addtolength\headsep{2mm}
\setlength\footskip{10mm}


%- typesetting code -------------------------------
\usepackage{listings}                                 

%\lstset{
    %frame=single,
    %breaklines=true,
%    postbreak=\raisebox{0ex}[0ex][0ex]{\ensuremath{\color{red}\hookrightarrow\space}}
%}

% estilo default de typesetting de código
\newstyle{.myclisting}{font-family:monospace;white-space:pre;margin:lex;padding:2ex}

% estilo para Haskell
%OBS: a versao original de Haskell considera como 
% keywords todas as definições do prelude.base
\lstdefinestyle{mystyle}{
  basicstyle={\tt},
  backgroundcolor={\color{Cyan}},
  keywordstyle={\bf \color{black}},
  identifierstyle={\color{blue}},
  commentstyle={\rm \it \color{DarkOrchid}},
  emphstyle={{\bf \color{Mahogany}}},
  xleftmargin=0pt,xrightmargin=0pt,
  keepspaces={true},
  showstringspaces=false,
  stepnumber=1,
  numbersep=5pt,
  numberstyle={\tt \footnotesize \color{white}},
  escapechar=!
}

\setenvclass{lstlisting}{myclisting}
\lstset{style=mystyle}

%\def\lst@lettertrue{\let\lst@ifletter\iffalse}

%\definecolor{LightGray}{gray}{0.90}

\lst@definelanguage{myHaskell}
                  { %language={Haskell},
                    style=mystyle,
                    morekeywords={if,then,else,case,class,data,default,deriving,
                      hiding,in,infix,infixl,infixr,import,instance,let,module,
                      newtype,of,qualified,type,where,do}
                  }

\lstnewenvironment{alg}[2]{
  \lstset{language=myHaskell,
    backgroundcolor={\color{#2}},
    basicstyle={\ttfamily},
    keywordstyle={\bf \color{black}},
    identifierstyle={\color{blue}},
    commentstyle={\rm \it \color{DarkOrchid}},
    emphstyle={{\bf \color{Mahogany}}},
    keepspaces={true}}
}
{}

\newcommand{\ina}{\lstinline[language=myHaskell]}

\setcounter{tocdepth}{3}

\title{Programa\c{c}\~ao em Haskell}

\author{Carlos Camarão\\
{\small Universidade Federal de Minas Gerais}\\
{\small Doutor em Ci\^encia da Computa\c{c}\~ao pela Universidade de Manchester, Inglaterra}\\ \\
Lucília Figueiredo\\
{\small Universidade Federal de Ouro Preto}\\
{\small Doutora em Ci\^encia da Computa\c{c}\~ao pela Universidade Federal de Minas Gerais}\\ \\
Rodrigo Ribeiro\\
{\small Universidade Federal de Ouro Preto}\\
{\small Doutor em Ci\^encia da Computa\c{c}\~ao pela Universidade Federal de Minas Gerais}\\ \\
Cristiano Vasconcellos\\
{\small Universidade Estadual de Santa Catarina}\\
{\small Doutor em Ci\^encia da Computa\c{c}\~ao pela Universidade Federal de Minas Gerais}}
%\date{Febuary 25, 2015}
\pagestyle{headings}

\htmlhead{\sffamily\bfseries{Programação em Haskell}}

\newcommand{\eqdef}{\overset{\text{def}}{=}}
\newcommand{\und}{\textbf}

\newcommand{\graybox}[1]{\begin{minipage}{\textwidth}\psboxit{box 0.95 setgray fill}{
  {\spbox{#1}}}\end{minipage}}

\makeindex
\setcounter{tocdepth}{3}

\setlength\textwidth{150mm}
\setlength\textheight{241mm}
\setlength\oddsidemargin{5.4mm}
\setlength\evensidemargin{5.4mm}
\addtolength\voffset{-5mm}
\addtolength\topmargin{-3mm}
\addtolength\headsep{2mm}
\setlength\footskip{10mm}

\pagestyle{fancyplain}
\addtolength\headwidth{\marginparsep}
\cfoot{}

\begin{document}

\maketitle

\begin{tabular}{l}
Direitos exclusivos\\
Copyright \copyright\ 2009 by Carlos Camar\~ao, Lucília Figueiredo, Rodrigo Ribeiro e Cristiano Vasconcellos
\end{tabular}

\'E permitida a duplica\c{c}\~ao ou reprodu\c{c}\~ao, no todo ou em
parte, sob quaisquer formas ou por quaisquer meios (eletr\^onico,
mec\^anico, grava\c{c}\~ao, fotoc\'opia, distribui\c{c}\~ao na Web ou
outros), desde que seja para fins n\~ao comerciais.

\pagestyle{fancy}
\pagenumbering{roman}
%\setcounter{page}{5}

\renewcommand{\chaptername}{Capítulo}
\renewcommand{\tablename}{Tabela}
\renewcommand{\figurename}{Figura}
\renewcommand{\contentsname}{Conteúdo}
\renewcommand{\indexname}{Índice}

\tableofcontents

\setcounter{secnumdepth}{-2}
\input{meta.keys}

\pagestyle{fancy}
\setcounter{secnumdepth}{10}
\pagenumbering{arabic}

\chapter{Prefácio}

Este é um livro sobre introdução a programação funcional em Haskell.
O livro é destinado a estudantes de graduação de cursos universitários
de Ciência da Computação e cursos correlatos. O objetivo do livro é
apresentar as ideias principais da programação funcional, como ela se
desenvolveu e vem se desenvolvendo a partir das ideias de programação
com tipos de dados polimórficos, tipos algébricos recursivos,
sobrecarga dependente de contexto e programação monádica.

Resolvemos escrever este livro por considerar que os livros atuais não
abordam ou enfocam os temas acima como achamos que deveria ser
feito. Em particular, consideramos que a definição e o uso de tipos
polimórficos, em particular de tipos de dados algébricos recursivos, é
essencial para facilitar na tarefa de programação, para definição de
funções que manipulam tais estruturas de dados, tipicamente listas e
árvores.

Na presença de recursos para definição de tipos polimórficos e de
tipos definidos recursivamente, o fato da linguagem ser estaticamente
tipada facilita enormemente a tarefa de programação. É essencial
apontar que é a possibilidade de definir tipos/estruturas de dados
recursivamente que é a característica mais importante, em vez da
possibilidade de definir funções recursivamente. É essa ideia básica
que motiva a primeira parte deste livro.

Procuramos também salientar a importância do mecanismo que permite a
definição e uso de funções sobrecarregadas para manipulação dessas
estruturas de dados polimórficas, como tipicamente as funções filter,
map e principalmente fold, mas também muitas outras.

A segunda parte do livro aborda a programação com funtores, funtores
aplicativos, e a programação monádica. ...

A terceira parte do livro aborda o tema de correção de programas. ...







%\input{introd}

%Considere a definição e uso de tipos recursivos em linguagens
%imperativas, como \C\ e \CMM.  ...

%[[ Aqui vem exemplos, talvez começando com o tipo listas, ou árvores
%binárias, e talvez depois outros... ]]

%\input{tipos-basicos}

%\input{tipos-algebricos}

%\input{verificacao-e-inferencia-de-tipos}
\chapter{Tipos}

\section{A regra básica}
\label{Regra-basica}

A regra:

  \[ \frac{f : A \rightarrow B \:\:\: x : A}{f\: x : B} \]

é a regra fundamental usada por um compilador na verificação da
correção e na inferência dos tipos usados em programas.

Note que existe uma correspondência básica entre tipos e termos
lógicos (termos da lógica proposicional)
\cite{Martin-Lof:Constructive-math-and-programming,Sorensen-Urzyczyn:2006:Lectures-Curry-Howard-Iso}.
  Se consideramos tipos ($A$, $B$, $A \rightarrow B$) como proposições
  lógicas, e expressões ($f$, $x$, $f \:x$) como provas
  (demonstrações) dessas proposições, a regra acima é a regra
  fundamental (chamada de {\em modus ponens\/}) usada em lógica para
  dedução de novas formas válidas a partir de formas válidas
  existentes.

A correspondência entre proposições lógicas e tipos, e entre provas
(demonstrações) e expressões, é conhecida como ``correspondência de
Curry-Howard'' (também comumente chamada de ``isomorfismo de
Curry-Howard'')
\cite{Martin-Lof:Constructive-math-and-programming,Sorensen-Urzyczyn:2006:Lectures-Curry-Howard-Iso}.
%Foi ....

No caso de linguagens monomórficas, esta é a regra mais fundamental, e
as demais podem ser consideradas como auxiliares. Por exemplo, para
deduzir que \ina{(not True)} tem tipo \ina{Bool}, precisamos apenas
considerar ou saber que \ina{not} tem tipo \ina{(Bool->Bool)} e
\ina{True} tem tipo \ina{Bool}. Essa consideração ou conhecimento usa
tipos existentes em um contexto, nesse caso tipos pré-definidos de
constantes e funções pré-definidas.

% (Falar de duplas/tuplas e currying?) 

\section{Polimorfismo paramétrico}

No caso de linguagens polimórficas, essa regra também deve ser
aplicada, mas após {\em instanciação\/} do tipo do domínio da função
para que fique igual ao tipo do argumento. No caso da inferência de
tipos, essa instanciação é feita por uma função de {\em unificação\/},
que calcula a substituição mínima que se pode aplicar para que o tipo
do domínio da função e o tipo do argumento se tornem iguais.

Mais detalhadamente o processo de inferência de tipos usa a seguinte
regra de aplicação, que vamos chamar de regra (\APL): para que uma
função (potencialmente polimórfica) {\tt $f$: $t$} possa ser aplicada
a um argumento {\tt $x$: $t'$}, devemos {\em unificar\/} $t$ com
$t_A\rightarrow t_B$ (ou seja, $t$ deve ter um tipo funcional) e $t_A$
com $t'$ (ou seja, o tipo do argumento deve ser unificado com o tipo
do domínio da função); o tipo resultante é o tipo $t_B$ resultante da
unificação. Isso vai ser explicado mais precisamente na seção
seguinte.

O processo de inferência de tipos de um compilador atribui a valores
que ocorrem em expressões tipos que são conhecidos e, sempre que um
tipo de um valor ainda não é conhecido, como por exemplo o tipo de um
parâmetro formal de uma função, uma nova variável de tipo, ainda não
usada, é (inicialmente) atribuída a esse valor. Depois disso, a
inferência de tipos consiste em determinar os tipos resultantes das
unificações especificadas na regra (\APL), que vai ser abordada mais
detalhadamente na seção seguinte.

\subsection{Unificação e Substitituição}

Dado um conjunto de variáveis $V$, um conjunto de constantes $C$ e um
conjunto de nomes de funções $F$, o conjunto de termos $T$ pode ser
definido pelo seguinte tipo algébrico recursivo:

\prog{\data\ T = Con K | Var V | App T T}
onde supomos que $K$ é um conjunto de constantes e $V$ de variáveis de
tipo.

Um tipo como $t_1 \cdots\ t_n$ é visto, nessa representação, como
 $\App\ t_1 (\ldots\ \App\ t_{n-1}\: t_n)$.
 
Uma substituição é uma função finita $S: V_0 \rightarrow T$ de
variáveis em termos, onde $V_0$ é o conjunto finito de variáveis que
ocorrem em $T$. A função \id\ é a função identidade, isto é, a função
definida por $\id\: x = x$.

Dado um conjunto finito de equações $E = \{ t_i=t_i'|i=1,\ldots,n\}$,
onde $t_i, t_i'$ são termos (valores de $T$), uma substituição
$\sigma$ é um unificador dos termos de $E$ se
$\sigma(t_i)=\sigma(t_i)'$, para $i=1,\ldots,n$.

Além disso, uma solução $\sigma$ é mais geral que outra $\sigma'$ se
existe uma substituição $\sigma_1$ tal que
$\sigma=\sigma_1\circ\sigma'$. Intuitivamente, um unificador mais
geral de um conjunto de equações $E$ é o modo mais simples de tornar
todas as equações do conjunto $E$ iguais.

O seguinte algoritmo retorna \Just\ $m$, onde $m$ é mapeamento que
representa o unificador mais geral para uma lista qualquer de pares de
tipo (cada par representando uma equação), se um unificador existir,
caso contrário retorna \Nothing\ ({\tt (.)} é o operador de composição
de funções e \ina{ocorreEm} testa ocorrência de variável em tipo):

\begin{center}
\begin{tabular}{l}
\begin{alg}{.9\textwidth}{white}
unif [] = id 
unif ((t, V v): eqs)          = unif ((V v, t): eqs)  
unif ((V v, t): eqs)                                  
  | t == V v                  = unif eqs              
  | v `ocorreEm` t            = Nothing               
  | otherwise                 = s' . s                
  where                                               
    s    = insert v t empty                           
    eqs' = ap s eqs                                   
    s'   = unif eqs'                                  
unif ((K k1, K k2): eqs)                              
  | k1 == k2                  = unif eqs              
  | otherwise                 = Nothing               
unif (App t1 t2, App t1' t2') = s' . s                
  where s  = unif t1 t1'                              
        s' = unif (app s t2) (app s t2')              
unif _                        = Nothing               
\end{alg}
\end{tabular}
\end{center}

As regras usadas na inferência de tipos de um compilador de uma
linguagem com polimorfismo paramétrico são baseadas em um {\em
  contexto de tipos\/}, que associam tipos a variáveis (constantes
podem ser consideradas como variáveis, com tipos a ela associados em
um contexto global). Contextos de tipos mapeiam variáveis a tipos (é
comum na literatura o uso de conjuntos de pares representados na forma
{\em variável : tipo\/}). Vamos usar, como é comum, a meta-variável
$\Gamma$ para representar contextos de tipos.

Tipos de variáveis podem ser monomórficos, ou {\em simples\/} ou
polimórficos, ou {\em quantificados\/}. Um tipo simples é formado por
uma constante ou uma variável, possivelmente aplicada a um ou mais
tipos simples. Ou seja, é da forma:
\[ T \texttt{ ::= } K \mid V \mid T\: \bar{T} \]

Usamos $t$ como para denotar valores do tipo $T$, $a$ ou $b$ para
denotar valores de tipo $V$ e $k$ para denotar valores do tipo $K$,
podendo essas variáveis ter possivelmente subscritos e aspas
simples. Usamos $\bar{X}$ para denotar sequências, possivelmente
vazias, de valores $X$, ou conjuntos de tais valores. Por exemplo,
$\bar{T}$ representa uma sequência de tipos, e um tipo polimórfico
$\sigma$, do tipo de conjuntos polimórficos $\Sigma$, é da forma
$\forall \bar{a}.\:t$; ou seja:
\[ \Sigma \texttt{ ::= } \forall \bar{V}.\: T \]

A notação $\Gamma, x:t$ representa o contexto $\Gamma'$ tal que
$\Gamma'(x') = t$ se $x'=x$, senão $\Gamma(x')$ (ou seja, $\Gamma'$ é
um contexto igual a $\Gamma$ mas $\Gamma'(x) = t$).

A função $\tv$ retorna o conjunto de variaveis livres de um tipo. Em
particular, $\tv(\forall\bar{a}.\:t) = \tv(t) - \bar{a}$.

A relação $\gen(t,\sigma,\Gamma)$ é a relação de generalização de $t$
para $\sigma$ em $\Gamma$, definida de forma que $\sigma =
\forall\bar{a}.\:t$, onde $\bar{a} = \tv(t) - \tv(\Gamma)$.

Um tipo $S(\sigma)$, ou simplesmente $S\sigma$, onde $S$a é uma
substituição tal que $S(a) = a$ se $a \not= a_i$, $S(a_i) = b_iq$,
para $i$ de 1 a $n$, representa o tipo resultante de substituir cada
$a_i$ por $b_i$, para $i$ de 1 a $n$. O tipo $S\sigma$ pode ser
denotado também por
 $\texttt{\symbol{91}}\bar{a}\texttt{ := }\bar{b}\texttt{\symbol{93}}\sigma$.

O algoritmo de inferência de tipos de uma linguagem como o núcleo da
linguagem \ML\ pode ser descrito usando as regras mostradas na Figura
\ref{Inf-tipos}, onde $\Gamma \vdash e: (t,S)$ significa que $e$ tem
tipo $t$ no contexto de tipos $\Gamma$. A substituição $S$ é usada
para unificação de tipos de expressões usadas em $e$. Esse algoritmo
foi definido por Robin Milner e Luís Damas \cite{DamasMilner82}, sendo
chamado de algoritmo $W$.

A regra (\VAR) retorna o tipo da variável no contexto de tipos
(supõe-se que um programa é um termo fechado, isto é, sem variáveis
{\em livres\/}, no sentido de $\lambda$-cálculo).

A regra (\APL) é a regra básica (ver seção \ref{Regra-basica}) da
inferência de tipos: usa unificação para tornar o tipo domínio da
função igual ao tipo do argumento.

A regra (\ABS) cria uma nova variável ($a$, na Figura \ref{Inf-tipos})
para o tipo do parâmetro, usando a substituição ($S$) obtida na
inferência do tipo do corpo da $\lambda$-abstração para especialização
do tipo do parâmetro. O tipo da $\lambda$-abstração é o tipo funcional
$S(a)\rightarrow t$, sendo $t$ o tipo inferido para a expressão $e$,
corpo da $\lambda$-abstração.

A regra (\LET) generaliza o tipo inferido para $e$ para obter o tipo
de $e'$, que é retornado como o tipo da expressão {\tt \llett\ $x$ =
  $e$ \iinn\ $e'$}. Note que a regra (\LET) é que permite a introdução
de tipos polimórficos; o uso de variáveis definidas na regra (\LET)
pode ser usada em contextos que requerem tipos distintos, como por
exemplo em:
\[ \texttt{\llett\ $f$ = $\lambda x.\:x$ \iinn\ ($f\: \True, f\:1$)} \]
onde $\True$ e $1$ têm tipos distintos (\Bool\ e \Integer), e o tipo
de $f$ pode ser instanciado (no algoritmo $W$, via unificação) para
\Bool $\rightarrow$ \Bool\ e para \Integer $\rightarrow$ \Integer.

\begin{figure}
  \[ \begin{array}{lr}
        \frac{\Gamma(x) = \forall\bar{a}.\:t  \:\:\:\: {\bar{b} \textrm{variáveis novas}}}
            {\Gamma \vdash x:\: (\symbol{91}\bar{a} {\texttt{:=}}\,\bar{b}\symbol{93}\,t\,,\:\iid)}
            & \texttt{(\VAR)} \\[.2cm]
        \frac{\Gamma \vdash f: \:(t,S)  \:\:\:\: S(\Gamma) \vdash e:\: (t'\,,\:S') \:\:\:\: a \textrm{variável nova}}
             {\Gamma \vdash f \: e: \:(S(a)\,,\:S_u \circ S' \circ S )}
            & S_u = \unif(\symbol{91}(t,\: t' \rightarrow a)\symbol{93}) \:\:\:\: {\tt (\APL)} \\[.2cm]
        \frac{\Gamma,x:a \vdash e:\:(t\,,\:S) \:\:\:\: a \textrm{variável nova}}
             {\Gamma \vdash \lambda x \texttt{->}\:e:\: (S(a)\texttt{->}\:t\,,\:S)} 
            & \texttt{(\ABS)} \\[.2cm]
        \frac{\Gamma \vdash e:\: (t\,,\:S) \:\:\: \gen(t,\sigma,S(\Gamma)) \:\:\: S(\Gamma),x:\sigma \vdash e':\: (t',S')}
             {\Gamma \vdash \llett\ x {\texttt{ = }} e\: \iinn\ e':\: (t'\,,\:S' \circ S)}
            & \texttt{(\LET)}
     \end{array}
  \]
\caption{Algoritmo W}
\label{Inf-tipos}
\end{figure}

A regra (\LET) (e a linguagem núcleo de \ML\ do algortimo $W$ definido
na Figura \ref{Inf-tipos}) não considera recursão: $x$ não pode
ocorrer em $e$ na expressão {\tt \llett\ $x$ = $e$ \iinn\ $e'$}.

%Para permitir definições recursivas, há duas possibilidades: a
%primeira é permitir a chamada {\em recursão polimórfica\/},
%inicialmente estudada por Mycroft \cite{Mycroft84}, mas isso leva a
%indecidibilidade da inferência de tipos (isso foi provado por Kfoury
%et al. \cite{KTU94}). O leitor interessado pode consultar também
%\cite{LuCarWFLP2001}.

No entanto, o uso do algoritmo definido na Figura \ref{Inf-tipos} não
é necessário para determinar o tipo de funções, na prática, uma vez
que ele é baseado fundamentalmente no uso do tipo de variáveis que
estão no contexto de tipos, pela introdução de novas variáveis de tipo
para o tipo de variáveis introduzidas em $\lambda$-abstrações, e pelo
uso de unificação em aplicações de funções a seus argumentos. Não é
necessário seguir todos os passos do algoritmo. Podemos usar a
seguinte técnica, que vamos chamar de
técnica-informal-de-inferência-de-tipos. Considere, por exemplo, a
tarefa de inferir o tipo da expressão:

\[ \lambda f.\:\lambda x.\:f\:(f\:x) \]\

Sabemos que o tipo dessa expressão é da forma:
\[ t_f \rightarrow t_x \rightarrow t_r \]
onde: \begin{tabular}[t]{l}
        $t_f$ (é uma variável que) representa o tipo de $f$,  \\
        $t_x$ (é (uma variável que) representa o tipo de $x$ e \\
        $t_r$ (é uma variável que representa) o tipo do resultado, $f\:(f\: x)$.
\end{tabular}
\newline
Introduzimos agora as informações que pudermos obter com os usos das variáveis 
$f$ e $x$ na expressão $f\:(f\:x)$, devido a unificação, que são:
\begin{itemize}
\item $f$ tem tipo funcional (pois $x$ é aplicado a $f$), ou seja, é da forma $t_1\rightarrow t_2$;
\item o tipo de $x$ é o mesmo do domínio de $f$, ou seja, $t_1 = t_x$;
\item o tipo do resultado de $f$ ($t_2$) é o mesmo do argumento de $f$ (pois o resultado, $f\:x$ é aplicado a $f$).
\end{itemize}.

Note: cada ocorrência de parâmetro de função --- i.e.~cada variável em
uma $\lambda$-abstração --- tem um tipo monomórfico (essa é uma
característica fundamental do sistema de tipos de \ML\ e \Haskell).

Usando essas informações, obtemos que o tipo de:
\[ \lambda f.\:\lambda x.\:f\:(f\:x) \]
é igual a (chamando $t_x$ simplesmente de $a$):
\[ (a \rightarrow a) \rightarrow a \rightarrow a \]

Essa técnica pode ser estendida para os casos de definições
recursivas, que envolvam mais de uma equação, (lembrando que
parâmetros têm tipos monomórficos e) considerando que:

\begin{itemize}
  \item uma equação {\tt $f\: p_1\: \ldots\: p_n$ = $e$} pode ser vista
    como tendo o tipo de uma $\lambda$-abstração: {\tt $\lambda p_1.\:
      \ldots\: \lambda p_n.\: e$}, sendo esse o tipo de $f$.
  \item toda definição:
    {\texttt{ \[\begin{array}[t]{ll}
                  f\: p_{11}\: \ldots\: p_{1n}\: & = \:e_1\\
                  \ldots & \\
                  f\: p_{m1}\: \ldots\: p_{mn}\: & = \:e_m
        \end{array}\]}}
      deve ser tal que os tipos de cada resultado $e_i$, assim como os
      tipos de cada parâmetro $p_{ij}$, para cada $i,j$
      ($i=1,\ldots,m$ e $j=1,\ldots,n$) são iguais: para isso os tipos
      de cada equação devem ser unificados (se não for possível
      unificar os tipos de cada equação, a definição não é bem
      tipada).
      
\end{itemize}

Por exemplo, considere a definição de \map:

\begin{center}
\begin{tabular}{l}
\begin{alg}{.9\textwidth}{white}
  map f []    = [] 
  map f (a:x) = f a : map f x
\end{alg}
\end{tabular}
\end{center}

Vamos calcular o tipo de \map\ inicialmente na primeira
equação. Poderíamos começar com $t_f \rightarrow t_l \rightarrow t_r$, sendo
$t_f$, onde:
\begin{quotation}
  \begin{tabular}{l}
        $t_f$ é (uma variável que representa) o tipo de $f$,  \\
        $t_x$ é (uma variável que representa) o tipo de {\tt []} e \\
        $t_r$ é (uma variável que representa) o tipo do resultado, também {\tt []}.
  \end{tabular}
 \end{quotation}
No entanto, como sabemos, pelo tipo de {\tt []} no contexto de tipos,
que {\tt []} tem tipo {\tt [$a$]}, sendo $a$ uma variável nova,
podemos considerar que o tipo de \map\ na primeira equação é:

{\ttfamily
  \[ t_f \rightarrow [a] \rightarrow [b] \]
}

Note que as ocorrências de {\tt []} têm tipos distintos, com variáveis
de tipo $a$ e $b$ novas, pois o tipo de {\tt []}, no contexto de
tipos, é polimórfico.

Na segunda equação, podemos atribuir a \ina{map} inicialmente o tipo:
$t_f \rightarrow t_l \rightarrow t_r$, sendo:
     $t_f$ variável que representa o tipo de $f$,  
     $t_l$ variável que representa o tipo de \ina{a:x} e
     $t_r$ variável que representa o tipo de \ina{f a:map f x}.
Temos:

\begin{itemize}
  \item sendo $t_l$ igual a {\tt [$a$]} (de acordo com o tipo de {\tt
    []} usado para esse parâmetro na primeira equação), o tipo de
    \ina{a} é igual a $a$ e o tipo de \ina{x} é igual a {\tt [$a$]}
    (pois o tipo de \ina{(:)} no contexto de tipos é igual a
    $a'\rightarrow \texttt{[$a'$]} \rightarrow \texttt{[$a'$]}$, para
    alguma variável nova $a'$, mas $a'$ deve ser igual a $a$, pois os
    tipos dos parâmetros nas duas equações devem ser iguais.

  \item $t_f$ deve ser um tipo funcional, pois \ina{f} é aplicado a
    \ina{a}, e o tipo do resultado de \ina{f} deve ser igual a $b$,
    igual ao tipo do elemento da lista {\tt [$b$]}: {\tt [$b$]} é o
    resultado de \ina{map f []} na primeira e na segunda equações, e
    igual ao tipo do resultado de \ina{map f x}.

\end{itemize}

Portanto, o tipo de \ina{map}, definido pelas duas equações acima, é:

\[ (a \rightarrow b) \rightarrow \texttt{[$a$]} \rightarrow \texttt{[$b$]}
\]

\subsection{Sobrecarga e Tipos restritos}
\label{Tipos-restritos}

Um valor sobrecarregado tem um tipo polimórfico restrito (ou
constrito) --- por exemplo, {\tt (==)} tem tipo
  \ina{Eq a => a -> a -> Bool}.

{\em Quando a sobrecarga é resolvida\/} (e apenas neste caso), devido
a instanciação do tipo polimórfico (via unificação), é preciso
verificar se há instâncias no contexto que satisfazem ao tipo do valor
sobrecarregado que foi instanciado:

\begin{itemize}
  \item se há apenas uma instância, a restrição é removida do tipo;
  \item se não ha nennhuma instância, ocorre um erro de tipo:
    insatisfazibilidade.
  \item se há duas ou mais instâncias, ocorre um erro de tipo: ambiguidade.
\end{itemize}

Por exemplo, a expressão {\tt 1 + '1'} usualmente gera um erro de
tipo, porque usualmente não há instância de \Num\ para o tipo \Char.

Em Haskell, infelizmente, não há, ainda, definição de quando uma
sobrecarga é resolvida, e esta condição (quando uma sobrecarga é
resolvida) é confundida com ambiguidade: em Haskell um tipo é ambíguo
se existe uma variável de tipo que ocorre em uma restrição e não
ocorre no tipo simples. Por exemplo, o tipo da expressão
\ina{(show . read)},
que é:
\begin{center}
\begin{tabular}{l}
\begin{alg}{.9\textwidth}{white}
  (Read a, Show a) => String -> String
\end{alg}
\end{tabular}
\end{center}
é considerado ambíguo, pois a variável de tipo \ina{a} ocorre em uma
restrição e não ocorre no tipo (simples) \ina{String -> String}.

Em Haskell, não há, assim, diferença entre insatisfazibilidade e
ambiguidade, nem entre sobrecarga resolvida e ambiguidade.

O teste de satisfazibilidade, que deve verificar se existe ou não uma
única instância que satisfaz a uma restrição, é em geral um problema
indecidível \cite{Smith91,VolpanoSmith91}. Existem várias opções de
compilação no GHC para restringir tipos polmórficos restritos
\cite{GHC}. No entanto, essas opcões poderiam ser evitadas por um
mecanismo baseado em uma medida do tamanho dos tipos que formam as
restrições, medida essa calculada durante o processo de testar
satisfazibilidade 
\cite{Ambig-and-cxt-dep-overloading-2013,Ambig-and-constrained-poly-2016}.

Quando há mais de uma equação em uma definição, as restrições de cada
equação devem ser consideradas (isto é, devem ser unidas) para gerar a
restrição final resultante da unificação de cada uma das
equações. Veja por exemplo o exercício resolvido \ref{Ex:merge}.

\section{Exercícios Resolvidos}

\begin{enumerate}

% \item Determine o tipo de $\lambda f.\:\lambda x.\:f\:(f\: x)$,
%   usando a técnica-informal-de-inferência-de-tipos.
%
%   {\em Solução\/}: Podemos ver que $f$ é uma função, digamos $a
%   \rightarrow b$ (pois $f$ é aplicada a $x$ e a $(f\:x)$), onde $a$ e
%   $b$ são variáveis novas. Podemos ver que $a$ é igual ao tipo de
%   $x$, e que o tipo de retorno de $f$ (igual a $b$) também é igual a
%   $a$, pois $f\:x$ (retorno de $f$) é também argumento de $f$.
%   Portanto, o tipo de $\lambda f.\:\lambda x.\:f\:(f\: x)$ é:
%
%     \[ (a \rightarrow a) \rightarrow a \rightarrow a \]

 \item \label{Ex:merge} Considere a definição de \ina{merge} a seguir:

\begin{center}
\begin{tabular}{l}
\begin{alg}{.9\textwidth}{white}
merge :: ([a], [a]) -> [a]
merge ([], y) = y                   -- (1)
merge (x, []) = x                   -- (2) 
merge (a:x, b:y)                    -- (3) 
  | a <= b    = a:merge (x, b:y)    -- (4)
  | otherwise = b:merge (a:x, y)    -- (5)
\end{alg}
\end{tabular}
\end{center}

Use a técnica-informal-de-inferência-de-tipos para determinar o tipo
de \ina{merge}.

{\it Solução\/}: O tipo de \ina{merge} é, inicialmente, igual a $t_p
\rightarrow t_r$, onde $t_p$ é o tipo do argumento e $t_r$ o tipo do
resultado.  Vamos determinar o tipo de cada equação, depois vamos
considerar que o tipo de cada equação é o mesmo, e para isso vamos
usar unificação, conservando (isto é, unindo) as restrições existentes
nos tipos de cada equação.

O tipo de \ina{merge} na linha \ina{1} (primeira equação) é
  {\tt ([$a$],$t_y$) -> $t_y$}.

O tipo de \ina{merge} na linha \ina{2} (segunda equação) é {\tt
  ($t_x$,[$b$]) -> $t_x$}.

Na linha \ina{3} (são duas equações neste caso, uma para cada guarda),
o tipo de \ina{merge} é {\tt \Ord\ $a$ => ([$a$],[$a$]) -> [$a$]};
  note que:
  \begin{itemize}
    \item a restrição {\tt \Ord\ $a$} é inserida devido ao uso do
      operador sobrecarregado {\tt <=} (na linha \ina{4});
    \item inicialmente o tipo de \ina{merge} na linha \ina{3} seria
      {\tt ([$a$],[$b$]) -> $t_r$}, mas, tanto devido à comparação
      \ina{a <= b} --- o tipo de {\tt <=} é
      \ina{Ord a => a -> a -> Bool},
      portanto o tipo de \ina{a} e de \ina{b} em \ina{a <= b} têm que
      ser iguais --- quanto devido ao fato de que os tipos de $t_r$
      nas linhas \ina{4} e \ina{5} têm que ser iguais ($t_r$ é igual
      {\tt [$a$]} na linha \ina{4} e igual a {\tt [$b$]} na linha
      \ina{5}, o tipo de \ina{merge} na última ``equação'' é
          {\tt \Ord\ $a$ => ([$a$],[$a$]) -> [$a$])}.
  \end{itemize}

  Unificando os tipos das equações, obtemos o tipo de \ina{merge}:

    \[ \texttt{\Ord\ $a$ => ([$a$],[$a$]) -> [$a$]} \]

\end{enumerate}

\section{Exercícios}

\begin{enumerate}

\item Determine o tipo de $\lambda x.\:\lambda y.\:\lambda z.\:x\: z
  (y\: z)$, usando a técnica-informal-de-inferência-de-tipos.

\item Use a técnica-informal-de-inferência-de-tipos para determinar o
  tipo da função \ina{either} definida abaixo:

  \begin{enumerate}

  \item Sendo o tipo \Either\ é definido como:
\begin{center}
\begin{tabular}{l}
\begin{alg}{.9\textwidth}{white}
  data Either a b = Left a | Right b
\end{alg}
\end{tabular}
\end{center}  
defina como é inferido o tipo de \ina{either}, usando a
técnica-informal-de-inferência-de-tipos, definido como:
\begin{center}
\begin{tabular}{l}
\begin{alg}{.9\textwidth}{white}
    either f g (Left x)  =  f x
    either f g (Right y) =  g y
\end{alg}
\end{tabular}
\end{center}  

  \end{enumerate}
  
\item Defina expressão ou função com tipo:

  \begin{enumerate}

  \item {\tt (\Ord\ a, \Show\ a) -> \String}

  \item {\tt ($a$ -> $b$ -> $c$) -> $b$ -> $a$ -> $c$}

  \item {\tt ($a$ -> $b$, $a$ -> $c$) -> $a$ -> ($b$,$c$)}

  \item {\tt ($a$ -> $b$, $b$ -> $d$) -> ($a$,$b$) -> ($c$,$d$)}

  \item {\tt ($b$ -> $c$) -> ($a$->$b$) -> $a$ -> $c$}
  
  \item {\tt [($a$, $b$)] -> ([$a$], [$b$])}

  \item {\tt ($a$ -> $b$ -> $a$) -> $a$ -> [$b$] -> [$a$]}

  \end{enumerate}
    
\end{enumerate}


%\input{funcoes-de-alta-ordem}

%\input{correcao}
  
\chapter{Funções de ordem superior}

Funções de ordem superior como \map\ e \filter\ permitem a
implementação de operações comuns em estruturas de dados: aplicar uma
função a todos os elementos da estrutura de dados e filtrar os
elementos que satisfazem a um predicado.

No entanto, estes são casos especiais de uma operação mais expressiva,
chamada de {\em fold} (ou {\em reduce\/}), que permite obter um
resultado pela aplicação de uma função binária aos elementos de uma
estrutura de dados para obtenção de um resultado final, a partir de um
valor inicial.

Vamos trabalhar nestas notas de aula com {\em fold\/}s sobre listas,
para as quais podemos ter \ina{foldr} (``fold right'', i.e. \ina{fold}
sobre os elementos da lista da direita para a esquerda, em outras
palavras, do último até o primeiro elemento da lista) e \ina{foldl}
(``fold left'', i.e. \ina{fold} dos elementos da lista da esquerda
para a direita, em outras palavras, do primeiro até o último
elemento).

\section{\foldr}

Consideremos primeiro \ina{foldr}. O tipo dessa função é:

\begin{center}
\begin{tabular}{l}
\begin{alg}{.5\textwidth}{white}
  (a -> b -> b) -> b -> [a] -> b}
\end{alg}
\end{tabular}
\end{center}
    
\ina{foldr f z} aplicado a uma lista {\tt [$x_1$, $\ldots$, $x_n$]}
fornece o resultado:

\begin{equation}
  x_1\: \symbol{96}f\,\symbol{96}\: (\ldots\: (x_{n-1}\: \symbol{96}f\,\symbol{96}\: (x_n\: \symbol{96}f\,\symbol{96}\: z))\ldots)
  \label{comportamento-foldr}
\end{equation}

Em outras palavras, o resultado de \ina{foldr f z} aplicado à lista
{\tt $x_1\:$:$\:$($\ldots$ ($x_n\:$:$\:$[]))} é obtido substituindo
{\tt []} por \ina{z} e {\tt (:)} por {\tt
  \symbol{96}\ina{f}\symbol{96}}.

O valor de tipo \ina{b} obtido em aplicações sucessivas de \ina{f} em
um {\em fold} é chamado de {\em acumulador\/}.

O resultado é calculado da direita para a esquerda, isto é, a primeira
aplicação é {\tt \ina{f} $x_n$ \ina{z}}, depois {\tt \ina{f} $x_{n-1}$
  $r_1$}, sendo $r_1$ o resultado obtido por essa primeira aplicação,
até {\tt \ina{f} $x_1$ $r_{n-1}$}, sendo esse o último e $r_{n-1}$ o
penúltimo resultado obtido com aplicações de $f$.

\subsection*{Exemplos}

Exemplos básicos de uso de \foldr\ são mostrados a seguir. Procure
exercitar, definindo você mesmo, antes de ver a definição.

Para isso, pense o seguinte: \foldr\ $f$ $z$ obtém um valor final
aplicando $f$ a partir do valor inicial $z$. É preciso definir $f$ e
$z$. Leve em conta que: $f$ recebe um elemento da lista, o resultado
(de tipo $r$) de fazer \foldr\ no restante da lista, e retorna um
valor (do mesmo tipo de $r$); $z$ é o valor retornado quando a lista é
vazia.

É útil também ter em mente como \foldr\ computa seu resultado, como
mostrado em (\ref{comportamento-foldr}).

\begin{enumerate}
\item soma de todos os elementos:

\begin{center}
\begin{tabular}{l}
\begin{alg}{.3\textwidth}{white}
  sum = foldr (+) 0
\end{alg}
\end{tabular}
\end{center}

\item multiplicação de todos os elementos:

\begin{center}
\begin{tabular}{l}
\begin{alg}{.3\textwidth}{white}
  prod = foldr (*) 1
\end{alg}
\end{tabular}
\end{center}

\item concatenação de todos os elementos (de uma lista de listas):

\begin{center}
\begin{tabular}{l}
\begin{alg}{.37\textwidth}{white}
  concat = foldr (++) []
\end{alg}
\end{tabular}
\end{center}

\item conjunção de todos os elementos: 

\begin{center}
\begin{tabular}{l}
\begin{alg}{.35\textwidth}{white}
  and = foldr (&&) True
\end{alg}
\end{tabular}
\end{center}

\item disjunção de todos os elementos: 

\begin{center}
\begin{tabular}{l}
\begin{alg}{.35\textwidth}{white}
  or = foldr (||) False
\end{alg}
\end{tabular}
\end{center}

\item máximo dos elementos (de uma lista não vazia):

\begin{center}
\begin{tabular}{l}
\begin{alg}{.5\textwidth}{white}
  maximum (a:x) = foldr max a x
\end{alg}
\end{tabular}
\end{center}

\item mínimo dos elementos (de uma lista não vazia):

\begin{center}
\begin{tabular}{l}
\begin{alg}{.5\textwidth}{white}
  minimum (a:x) = foldr min a x
\end{alg}
\end{tabular}
\end{center}

\item comprimento (número de elementos):

\begin{center}
\begin{tabular}{l}
\begin{alg}{.45\textwidth}{white}
  len = foldr (\ _ -> (+1)) 0
\end{alg}
\end{tabular}
\end{center}

\item \map:

\begin{center}
\begin{tabular}{l}
\begin{alg}{.4\textwidth}{white}
  map f = foldr (:) []
\end{alg}
\end{tabular}
\end{center}
  
\item \filter:

\begin{center}
\begin{tabular}{l}
\begin{alg}{.8\textwidth}{white}
  filter p = foldr (\ a -> if p a then (a:) else id) []
\end{alg}
\end{tabular}
\end{center}

\end{enumerate}

Mais exemplos são mostrados a seguir.

\begin{enumerate}

\item teste de pertinência a lista:

\begin{center}
\begin{tabular}{l}
\begin{alg}{.2\textwidth}{white}
    elem a = foldr ((||) . (==a)) False
\end{alg}
\end{tabular}
\end{center}
  
\item ordenação por inserção: 

\begin{center}
\begin{tabular}{l}
\begin{alg}{.5\textwidth}{white}
insertionSort = foldr ins []
  ins a []      = [a]
  ins a (b:x)
    | a <= b    = a:b:x
    | otherwise = b: ins a x
\end{alg}
\end{tabular}
\end{center}

\item inserção de valor (\ina{k}) entre dois elementos consecutivos de
  uma lista; na definição de \ina{intersperse} mostrada abaixo, um
  valor booleano é usado para não inserir \ina{k} após o último
  elemento:

\begin{center}
\begin{tabular}{l}
\begin{alg}{.8\textwidth}{white}
intersperse k = fst . foldr consIfTrue ([],(k,False))
  where consIfTrue a (x,(k,False)) = (a:x  ,(k,True))
        consIfTrue a (x,(k,b    )) = (a:k:x,(k,b   ))
\end{alg}
\end{tabular}
\end{center}

\end{enumerate}

\section{\foldl}

\ina{foldl f z} aplicado a uma lista {\tt [$x_1$, $\ldots$, $x_n$]} fornece o
resultado:
  \[ \texttt{(\ldots ((z \symbol{96}\ina{f}\symbol{96} $x_1$) \symbol{96}\ina{f}\symbol{96} $x_2$) \ldots \symbol{96}\ina{f}\symbol{96} $x_n$)}\]

O tipo de \ina{foldl} é:

\begin{center}
\begin{tabular}{l}
\begin{alg}{.5\textwidth}{white}
  (b -> a -> b) -> b -> [a] -> b
\end{alg}
\end{tabular}
\end{center}

A função \ina{f} em \ina{foldl f}, ao contrário de \ina{foldr}, recebe
o ``acumulador'' primeiro, e depois o elemento da lista.

\fold\ é útil em casos (não muito comuns) em que o primeiro argumento
é não estrito no segundo argumento, ou quando se deseja inverter a
ordem dos elementos da lista.

Exemplos de uso de \foldl:

\begin{enumerate}

\item inversão da ordem dos elementos da lista:

\begin{center}
\begin{tabular}{l}
\begin{alg}{.2\textwidth}{white}
    reverse = foldl (flip (:)) []
\end{alg}
\end{tabular}
\end{center}

\item subtração de todos elementos de lista a valor:

\begin{center}
\begin{tabular}{l}
\begin{alg}{.2\textwidth}{white}
  subtraiTodosDe n = foldl (-) n
\end{alg}
\end{tabular}
\end{center}

\end{enumerate}

\section{Ineficiência da complexidade de espaço da avaliação preguiçosa de \foldl}

Infelizmente, ao contrário da complexidade de tempo, a complexidade de
espaço da estratégia de avaliação preguiçosa não é ótima. Por exemplo,
a quantidade de espaço usada para obter o resultado de:

\[ \texttt{\foldl\ $f$ $z$ $x$} \]
é $O(n)$, onde $n$ é o tamanho da lista, quando $f$ é estrita.  A
complexidade de espaço usada para obter o resultado de
$\texttt{\foldl\ $f$ $z$ $x$}$ é também $O(n)$ quando $f$ é
estrita. Se a lista for grande, isso poderá acarretar problema de
falta de espaço suficiente para obter o resultado.

No entanto, no caso de \foldl, a complexidade de espaço usada pela
estratégia de avaliação estrita é $O(1)$. Isso ocorre, no cálculo de
$\texttt{\foldl\ $f$ $z$ $x$}$, sendo {\tt $x$ $=$
  [$x_1$,$\ldots$,$x_n$]}, porque o resultado $r_i$ de $f\:z\:x_i$ é
calculado imediatamente, em vez de ser salvo para ser calculado
posteriormente, no cálculo de $f\: r_1\: x_{i+1}$, para
$i=1,\ldots,n$, e isso evita que espaço de memória tenha que ser
alocado a cada $i$ de $1$ até $n$.

No caso de \foldr, a complexidade de espaço da avaliação preguiçosa
não pode ser melhorada (de $O(n)$ para uma complexidade menor) porque
a construção dos resultados tem que ser realizada a partir o primeiro
elemento da lista e terminar com o último elemento da lista (e isso
envolve um custo $O(n)$), de modo a permitir que os resultados sejam
calculados do último elemento da lista até o primeiro.

\section{\foldr\ versus \foldl}
  
Quando analisamos a eficiência, em termos de tempo e espaço, de
\foldr\ e \foldl, por exemplo para decidir quando usar uma ou outra
função, podemos chegar a conclusões interessantes.

Para isso, consideremos \foldr\ primeiro:

{\ttfamily
\[ \begin{array}{lllllllll}
      \foldr\ f\: z\:\: ( &\!\!\!\! x_1 & :   & (\ldots & :   & (x_n & :   & \texttt{[\,]}) & \ldots)) = \\ 
                          &\!\!\!\! x_1 & `f` & (\ldots & `f` & (x_n & `f` & z)  & \ldots)
\end{array}
\]}

Note:

\begin{enumerate}
    \item Se $f\:e$ não for estrita (isto é, se o resultado da
      avaliação de $e_1$ for suficiente para estabelecer o resultado
      de {\tt $f\:e_1\:e_2$}, ou seja, se a avaliação de $e_2$ puder
      não ser necessária para estabelecer o resultado de {\tt $e_1$
        `$f$` $e_2$}), a complexidade de tempo do cálculo de {\tt
        (\ina{foldr} $f$ $z$ $x$)} é sub-linear, pois o número de
      valores de $x$, a partir de $x_1$, necessários para que $f$
      estabeleça um resultado, será menor ou igual a $n$.

      Por exemplo, \ina{foldr (&&) True [False..]}  é igual a
      \ina{False} (apesar de \ina{[False..]}  ser infinita), e o tempo
      necessário para avaliação é constante (esse é o tempo necessário
      para o cálculo de \ina{False && l}, podendo \ina{l} ser qualquer
      lista, finita ou infinita).

      O mesmo acontece para a complexidade de espaço.

    \item Se, no entanto, se $f$ e $f\:e$ forem estritas (isto é, se o
      resultado da avaliação de $e_1$ e de $e_2$ forem necessários
      para a avaliação de {\tt $e_1$ `$f$` $e_2$}), a complexidade de
      tempo do cálculo de \ina{foldr} $f$ $z$ $x$ (onde $x$ é igual a
      \texttt{($x_1$ : ($\ldots$ : ($x_n$ : []) $\ldots$))}) é $O(n)$,
      onde $n$ é o tamanho de $x$.

      O mesmo acontece para a complexidade de espaço.

    \item No caso de \foldl, a função $f\:z e$ deve ser não-estrita no
      segundo argumento ($e$): i.e.~o resultado de $f\:z\:e$ deve
      poder ser obtido sem necessidade de avalizar $z$). Lembre-se:

      \[ \foldl\ f\: z\: ( x_1 : (\ldots : (x_n :  \texttt{[\,]}) \ldots)) = 
       \: f ( \ldots (f ( f\: z\: x_1 )\: x_2)\: \ldots\: x_n) \]

      Por exemplo, a avaliação de \ina{foldl (&&) True [False..]}  não
      termina, porque \ina{(&&)} é não-estrita no primeiro argumento
      (faz casamento de padrão no primeiro argumento) e não no
      segundo. Ou seja, é preciso o resultado de cada conjunção
      precisa ser obtido para obtençao da próxima conjunção (o que
      requer o processamento de uma lista infinita de valores iguais a
      \False.
      
\end{enumerate}

\section{Exercícios}


\begin{enumerate}

\item Escreva, usando \foldl\ ou \foldr\, uma função que recebe uma
  lista de cadeias de caracteres (valores do tipo \String) e retorna
  uma cadeia de caracteres que contém os 3 primeiros caracteres de
  cada cadeia.

  Por exemplo, ao receber {\tt ["Abcde", "1Abcde", "12Abcde", "123Abcde"]}
  deve retornar {\tt \symbol{34}Abc1Ab12A123\symbol{34}}.

\item Escreva, usando \foldr\ ou \foldl, uma função que recebe uma
  lista de pessoas (valores de tipo \Pessoa, veja definição a seguir)
  e retorna a soma das idades (valores do campo \idade) de todos os
  elementos da lista.

\begin{center}
\begin{tabular}{l}
\begin{alg}{.5\textwidth}{white}
   data Pessoa = Pessoa {nome::Nome, idade::Idade, id::RG}
   type Nome   = String
   type Idade  = Integer
   type RG     = String
\end{alg}
\end{tabular}
\end{center}

\item Escreva, usando \foldr\ ou \foldl, uma função que recebe uma
  lista de valores de tipo Item, da definição acima, e retorna o nome
  da pessoa mais nova da lista.

\item Escreva, usando \foldl\ ou \foldr, uma função que recebe uma
  lista de cadeias de caracteres (valores do tipo \String) e retorna
  uma cadeia de caracteres que contém os 3 primeiros caracteres de
  cada cadeia removidos se não forem letras, ou com as letras em caixa
  alta se forem letras, e com os demais caracteres depois dos 3
  primeiros sem alteração.

  Por exemplo, ao receber {\tt ["Abcde", "1Abcde", "12Abcde", "123Abcde"]}
  deve retornar {\tt \symbol{34}ABCdeABcdeAbcdeAbcde\symbol{34}}.

\item Explique porque \foldr\ $f$ $x$ pode não percorrer toda a lista
  $x$, ao passo que toda a lista x é sempre percorrida, no caso de
  foldl.

\item A função \remdups\ remove elementos iguais adjacentes de uma
  lista, conservando só um dos elementos.

   Por exemplo, {\tt \remdups\ [1,2,2,3,3,3,1,1] = [1,2,3,1]}.

   Defina \remdups\ usando \foldr\ ou \foldl. 

\end{enumerate}

%\end{document}


\chapter{Programação com Ações e Programação Monádica}
\label{Monadas}

\section{Introdução}

A solução adotada por \Haskell\ para permitir o uso de ações com
efeito colateral é explicada a seguir. Chamamos de {\em ação\/} uma
construção da linguagem cuja execução provoca uma ``mudança de
estado'', como modificar o valor armazenado em uma variável ou
realizar alguma operação de entrada e saída.

Observação: {\em Ação com efeito colateral\/} é um nome mais ortodoxo,
uma vez que a palavra {\em ação\/} é usada em Haskell para denotar
qualquer valor monádico --- isto é, qualquer valor de tipo (monádico)
$m$ $a$, para qualquer $a$ e qualquer $m$ definido como instância da
classe \ina{Monad}. No entanto, o uso do termo ``ação'' para um valor
desse tipo coloca uma qualificação indevida em valores monádicos: se
não provocam efeito colateral, esses valores são valores como outros
quaisquer. Portanto, usamos consistentemente ``ação'' com o sentido
que é usualmente em computação o sentido de ``ação com efeito
colateral''.

A solução adotada em Haskell é baseada na seguinte ideia básica:

  \begin{quotation}
     {\bf Um construtor é usado para distinguir uma ação do resultado
       que ela produz.}
  \end{quotation}

Em Haskell, valores de tipo {\tt \IO\ $a$}, para um tipo \ina{a}
qualquer, s\~ao a\c{c}\~oes (com efeito colateral). O tipo que é
instância de \ina{a} é o tipo do valor retornado pela execução da
ação.

Valores do tipo \ina{ST s a} também são ações. Neste tipo, a variável
de tipo \ina{a} pode ser instanciada, como qualquer variável de tipo,
mas a variável de tipo \ina{s} tem uma característica especial: ela
não pode ser instanciada em argumentos de \ina{runST} (veja explicação
a seguir).

Nesses argumentos, a variável \ina{s} existe apenas para identificar
ou dar um nome a, digamos, um ``fluxo de execução''. O termo ``fluxo
de execução'' é apenas um nome dado ao argumento de \ina{runST} (uma
sequência de ações quaisquer), de tipo \ina{ST s a} (para algum
\ina{a} e, como vamos ver, para um \ina{s} que não pode ser
instanciado). Tipos (\ina{STRef s a}) de variáveis que podem ser
modificadas são ``etiquetadas com esse \ina{s}''. Isso é feito para
que ações que modificam o valor dessas variáveis só possam existir
nesse argumento (diz-se ``nesse fluxo de execução''). Assim, uma
variável etiquetada não pode ser usada novamente em outro fluxo de
execução. No entanto, o resultado obtido na execução desse argumento
(nesse ``fluxo de execução'') deve poder ser usado em outro fluxo de
execução.

É possível em Haskell, dessa forma, modificar o valor armazenado em
uma ``variável'', no sentido de ``variável'' em linguagens e programas
imperativos. Em Haskell, tais entidades são chamadas de referências e
devem ter tipo \ina{STRef s a}, para algum \ina{a} e para algum
\ina{s}. (O tipo \ina{IORef} também pode ser usado para modificação do
valor armazenado em variáveis, mas não carrega etiqueta, e por isso o
valor armazenado não pode ``sair'' da mônada \ina{IO}). O tipo
\ina{STRef s a} indica que a referência (variável, no sentido
imperativo) é local ao argumento de \ina{runST} (diz-se ``é local a um
determinado 'fluxo de execução'); essa etiqueta não pode ``sair'' da
mônada \ina{ST}, ou seja, não pode ser usada em nenhum outro argumento
a \ina{runST}.  As operações abaixo, que manipulam valores de tipo
\ina{STRef s a}, têm resultado de tipo \ina{ST s a} (para algum
\ina{a},\ina{s}):

\begin{center}
\begin{tabular}{l}
\begin{alg}{.5\textwidth}{white}
newSTRef :: a -> ST s (STRef s a)
readSTRef :: STRef s a -> ST s a
writeSTRef :: STRef s a -> a -> ST s ()
\end{alg}
\end{tabular}
\end{center}

A função \ina{newSTRef} recebe um valor de um tipo $a$ e retorna uma
ação que, quando executada, retorna uma referência para esse valor.

A função \ina{readSTRef} recebe uma referência a um valor de tipo
\ina{a} e retorna uma ação que, quando executada, retorna o valor
armazenado no endereço denotado por essa referência.

A função \ina{writeSTRef} recebe uma referência a um valor de tipo \ina{a}
e um valor de tipo \ina{a} e retorna uma ação que, quando executada,
armazena esse valor no endereço denotado pela referência.

Agora você poderá entender precisamente o que temos dito. O único modo
de usar um valor produzido pela execução de uma ação de tipo \ina{ST s a} 
(para algum \ina{a}) em um contexto que não envolve o construtor
\ina{ST}, é por meio do uso da função \ina{runST}, que tem um tipo
peculiar:

\prog{\ina{(forall s. ST s a) -> a}}

Esse é um tipo diferente dos tipos que vimos até agora. É chamado de
um tipo polimórfico de {\em grau 2\/} (em inglês, {\em rank 2\/}). Em
todos os tipos que vimos até agora, o quantificador \ina{forall}
ocorria externamente, antes do tipo não quantificado. Tais tipos têm
{\em grau 1\/}. Um tipo de grau 2, como o de \ina{runST}, requer que o
argumento tenha um tipo no qual o estado (\ina{s}) não é instanciado.

Considere por exemplo a diferença entre 
  $t_1 = $ \ina{forall a. a->Int} e 
  $t_2 = $ \ina{(forall a. a)->Int}. 
O único valor de tipo \ina{forall a. a} é o que denota não-terminação
($\perp$). No entanto, a variável de tipo \ina{a} pode ser instanciada
para fornecer várias funções com tipos que são instâncias de $t_2$
(por exemplo, o tipo de \ina{Int -> Int} é instância de $t_2$).

O tipo de \ina{runST} evita o uso de \ina{runST} em expressões como: 

\begin{center}
\begin{tabular}{l}
\begin{alg}{.5\textwidth}{white}
let t = runST (newSTRef True)
    f = runST (writeSTRef t False)
in runST (readSTRef t)
\end{alg}
\end{tabular}
\end{center}

Essa expressão não é tipável: note que \ina{newSTRef True} tem tipo
\ina{ST s (STRef s Bool)}. A expressão \ina{runST (newSTRef Bool)}
teria então que ter tipo \ina{STRef s Bool}, ou seja, \ina{a} teria
que ser unificado com \ina{STRef s Bool}. No entanto, isso modificaria
o tipo \ina{\ST s a}, requerido por \ina{runST}, porque o tipo do
resultado de \ina{runST (STRef s Bool)} passa a depender de \ina{s} (é
comum dizer: a variável de tipo $s$ não pode ``escapar'' para o tipo
do resultado).

A presença da variável \ina{s} em um valor de tipo \ina{STRef s t}
(para um \ina{t} qualquer) explica porque existem dois tipos distintos
de referências em Haskell, construídas com \ina{STRef} e com
\ina{IORef}. O construtor \ina{IORef} não tem tal variável de estado;
um valor é construído apenas com um valor do tipo que desejado (para
qualquer \ina{t}, o tipo \ina{IORef t} é um tipo de referência para
valor de tipo \ina{t}). Em vez de existir uma condição de uso de
\ina{s} (que requer que \ina{s} não possa ser instanciada e não possa,
assim, ser usada em tipo de outro argumento para \ina{runST},
requer-se que não haja, ou pelo menos não seja usada em geral, nenhuma
função que ``saia da mônada'' \ina{IO} (ou seja, requer-se que não
haja ou não seja usada em geral nenhuma função como
\ina{unsafePerformIO}).

O exercício resolvido \ref{Ex:uso-ST} ilustra o uso da mônada
\ina{ST}.

Para ações de entrada e saída de dados, é usado o construtor \ina{IO};
por exemplo, a ação de ler um caractere, \ina{getChar}, tem tipo
\ina{IO Char}. Como já mencionado, o uso do construtor \ina{IO} indica
que se trata de uma ação, e distingue a ação do resultado obtido pela
execução da ação, que é um valor de tipo \ina{Char}.

Uma ação de entrada e saída pode ser executada, para se obter o valor
resultante de sua execução, mas:

  \begin{quotation}
    {\bf Não é possível definir função que recebe ação de tipo \ina{IO t}, 
         para um \ina{t} qualquer, e retorna valor de tipo \ina{t}.}
  \end{quotation}

Não se pode definir, por exemplo, uma função de tipo {\tt
  \IO\ \Char\ -> \Char}.  Note no entanto que existe uma função,
disponível na versão de Haskell implementada pelo compilador GHC
\cite{GHC}, que tem tipo \ina{IO a -> a}: a função
\ina{unsafePerformIO}.  Mas, como o próprio nome indica, seu uso não é
em geral seguro, ela só deve ser usada em circunstâncias especiais, em
que a mudança de estado provocada pela ação de entrada ou saída não
tem efeito indesejado. O uso de uma função como \ina{unsafePerformIO}
pode destruir a relevância da distinção entre a ação e o valor
retornado em sua execução.

As condições sobre o uso de ações em Haskell são essenciais para
garantir que a linguagem seja uma linguagem funcional pura, com {\em
  transparência referencial}.

% Essas condições, chamadas de CLFP ({\em condições para linguagem
%  funcional ser pura\/}) são: i) distinção entre o tipo de uma ação e
%  o resultado de sua execução --- em Haskell, essa distinção é feita
%  simplesmente com introdução de um construtor no tipo da ação ---, e
%  ii) não existência de função que recebe ação e retorna valor que
%  não envolve ação. ...

Em Haskell, a função \ina{main}, que representa o efeito de executar o
programa resultante de uma compilação, é uma ação, de tipo \ina{IO()}
(o parâmetro \ina{()} do construtor \ina{IO} indica que não há valor
útil retornado como resultado da execução dessa ação). A execução de
toda ação em um programa Haskell é iniciada a partir da execução da
ação definida na função \ina{main}.

%O fato de ações serem manipuladas em Haskell por meio do uso de
%funções monádicas (\return\ e {\tt >>=}) é uma questão secundária,
%embora importante. A questão principal, mais básica, é a explicitada
%pela CLFP acima.

%As únicas formas de manipular ações em Haskell são: i) executar uma
%ação, possivelmente obtendo o resultado dessa execução, e ii) compor
%uma ação $a_1$ sequencialmente com outra ação $a_2$ (a execução da
%ação composta consiste em executar $a_1$ e em seguida $a_2$). A
%execução da ação $a_2$ pode usar o resultado retornado pela execução
%de $a_1$.

Existem ações básicas de leitura e escrita, como por exemplo
\ina{getChar} e \ina{putChar} para leitura e escrita de um caractere,
e \ina{getLine} e \ina{putStr} para leitura e escrita de uma lista de
caracteres, que serão abordadas na seção \ref{Monada-IO}.

Na seção seguinte é apresentada uma introdução a mônadas em Haskell.
Em Haskell, ações e valores monádicos em geral são usados como
argumentos de ``combinadores'' (funções de ordem superior, não locais
a outras funções, geralmente usadas na construção e combinação de
valores de um determinado tipo). Funções de ordem superior proveem uma
maneira de estruturar programas, abordada na seção seguinte.

\section{Mônadas e estruturação de programas}
\label{Monadas-e-estruturacao-de-programas}

Existem dois {\em combinadores monádicos\/} principais, \ina{return} e
\ina{>>=}. É comum dizer: \ina{return} faz com que se entre em uma
mônada, e \ina{>>=} (costuma-se dizer, em inglês, {\em bind\/})
permite sequenciar ações monádicas. Esses combinadores constituem a
ferramenta principal usada na manipulação de ações e de valores
monádicos em geral em Haskell.

%Para mônadas que são ações de entrada e saída, não há como ``sair da
%mônada'', isto é, obter um valor não monádico --- mais precisamente,
%não é possível obter valor de algum tipo $t$ a partir de um valor
%monádico \IO\ $t$ (no entanto, já falamos anteriormente de
%\unsafePerformIO).

%Para mônadas que não são ações, no entanto, é possível ``sair da
%mônada'', isto é, simplesmente avaliar a expressão monádica para obter
%o valor resultante, não monádico.

Em Haskell, uma mônada é uma classe definida como a seguir:

\begin{center}
\begin{tabular}{l}
\begin{alg}{.5\textwidth}{white}
class Monad m where
  (>>=) :: m a -> (a -> m b) -> m b
  (>>) :: m a -> m b -> m b
  return :: a -> m a
  fail :: String -> m a
\end{alg}
\end{tabular}
\end{center}

Dado um tipo monádico (instância da classe \ina{Monad}) \ina{m t},
vamos chamar de \ina{t} o tipo-alvo do tipo monádico. Valor monádico é
um valor de tipo monádico e expressão monádica é uma expressão de tipo
monádico.

Em qualquer instância da classe \ina{Monad}, a função \ina{(>>=)}
compõe sequencialmemte dois valores monádicos, sendo a combinação
feita de modo que o valor resultante da avaliação (ou execução, no
caso de ações) do primeiro valor monádico é passado como argumento do
segundo.

A função \ina{(>>)} é semelhante: compõe sequencialmente dois valores
monádicos, mas a combinação é feita de modo que o valor resultante da
avaliação (ou execução, no caso de ações) do primeiro valor monádico é
descartado (em vez de ser passado para o segundo). A função \ina{(>>)}
pode ser facilmente definida em termos de \ina{(>>=)}:

\begin{center}
\begin{tabular}{l}
\begin{alg}{.5\textwidth}{white}
(m >> m') = m >>= \ _ -> m'
\end{alg}
\end{tabular}
\end{center}

A função \ina{return} apenas transforma o argumento em um valor
monádico, em geral simplesmente etiquetando o argumento com um
construtor do tipo-alvo do tipo monádico.

A função \ina{fail} receve uma cadeia de caracteres e retorna um valor
de tipo monádico que indica uma situação de erro. Essa função não é
parte da definição matemática de mônada, é usada com a notação
\ddo\ (veja seção \ref{Notacao-do} a seguir), quando ocorre falha em
casamento de padrões.

\subsection{Exercícios Resolvidos}

\begin{enumerate}

\item Escreva uma função \ina{sequence_} que recebe uma lista de ações
  e retorne uma ação que consiste em executá-las sequencialmente, uma
  após a outra, desprezando o resultado obtido com a sua execução. 

{\em Solução\/}: 

\begin{center}
\begin{tabular}{l}
\begin{alg}{.5\textwidth}{white}
sequence_ :: Monad m => [m a] -> m ()
sequence_ = foldr (>>) (return ())
\end{alg}
\end{tabular}
\end{center}

O uso de \ina{_} como sufixo do nome de uma função é usado em geral em
Haskell para indicar que os resutado da ação é \ina{()} (sendo
desconsiderados eventuais resultados de argumentos que são ações).

\item Escreva uma função \ina{for_} que recebe um inteiro $n$ e uma
  ação $m$ e retorna uma ação que consiste em executar $m$ exatamente
  $n$ vezes, desconsiderando os resultados.

{\em Solução\/}: 

\begin{center}
\begin{tabular}{l}
\begin{alg}{.5\textwidth}{white}
for_ :: Monad m => Int -> m a -> m ()
for_ n = sequence_  . replicate n
\end{alg}
\end{tabular}
\end{center}

\item Escreva uma função \ina{sequence} que recebe uma lista de ações
  e retorne uma ação que consiste em executá-las sequencialmente, uma
  após a outra, retornando a lista dos resultados obtidos com a sua
  execução.

\begin{center}
\begin{tabular}{l}
\begin{alg}{.5\textwidth}{white}
sequence :: Monad m => [m a] -> m [a]
sequence = foldr consa (return [])
  where consa ma m = do {a <- ma; x <- m; return (a:x)}
\end{alg}
\end{tabular}
\end{center}

\item Escreva função \ina{mapM_} que recebe uma lista de funções, cada
  uma delas retornando uma ação, e retorne uma ação que consiste em
  executar as ações resultantes de aplicar á-las sequencialmente, uma
  após a outra, retornando a lista dos resultados obtidos com a sua
  execução.

\begin{center}
\begin{tabular}{l}
\begin{alg}{.5\textwidth}{white}
sequence :: Monad m => [m a] -> m [a]
sequence = foldr consa (return [])
  where consa ma m = do {a <- ma; x <- m; return (a:x)}
\end{alg}
\end{tabular}
\end{center}
The two new map functions are:

\begin{center}
\begin{tabular}{l}
\begin{alg}{.5\textwidth}{white}
mapM_ f = sequence_ . map f
mapM f = sequence . map f

mapM :: (Traversable t, Monad m) => (a -> m b) -> t a -> m (t b)
\end{alg}
\end{tabular}
\end{center}

%Map each element of a structure to a monadic action, evaluate these
%actions from left to right, and collect the results. For a version
%that ignores the results see mapM_.

\begin{center}
\begin{tabular}{l}
\begin{alg}{.5\textwidth}{white}
mapM_ :: (Foldable t, Monad m) => (a -> m b) -> t a -> m ()
\end{alg}
\end{tabular}
\end{center}

\end{enumerate}

\section{Classes \ina{Foldable}, \ina{Traversable}}
\label{Foldable-Traversable}

\section{Classes \ina{Functor}, \ina{Applicative}}
\label{Applicative}

\section{Funtores}
\label{Funtores}

Funtores em Haskell são construtores de tipos que são instâncias da
classe \ina{Functor}, definida no \ina{Prelude}, para os quais é
definida uma função, chamada em Haskell de $\fmap$:

\begin{center}
\begin{tabular}{l}
\begin{alg}{.5\textwidth}{white}
class Functor f where
  fmap :: (a -> b) -> f a -> f b
\end{alg}
\end{tabular}
\end{center}

Para cada construtor de tipos \ina{f}, \ina{fmap g} aplica a função
\ina{g} a todos os elementos da estrutura de dados construída por meio
de \ina{f}. Note que \ina{f} é um construtor de tipos.
% (veja seção \ref{Kinds}).

Por exemplo, as funções \ina{map} e \ina{mapTree} com os tipos a
seguir devem ser definidas de modo a respectivamente aplicar uma
função fornecida como argumento a cada um dos elementos de uma lista
ou árvore:

\begin{center}
\begin{tabular}{l}
\begin{alg}{.5\textwidth}{white}
map :: (a -> b) -> [a] -> [b]
mapTree :: (a -> b) -> Tree a -> Tree b
\end{alg}
\end{tabular}
\end{center}

A função \ina{map} é uma instância de \ina{fmap} para listas (i.e.~uma
instância na qual \ina{f} é o construtor de tipos \ina{[]}) e,
analogamente, \ina{mapTree} é uma instância de \ina{fmap} para árvores
(i.e.~uma instância na qual \ina{f} é igual ao construtor de tipos
\ina{Tree}).  O tipo \ina{Tree} deve, é claro, ser visível para que
\ina{Tree} possa ser usado; a definição pode ser como a seguir:

\begin{center}
\begin{tabular}{l}
\begin{alg}{.5\textwidth}{white}
data Tree a = Folha a | Nodo (Tree a) (Tree a)
\end{alg}
\end{tabular}
\end{center}

As definições podem ser feitas como a seguir:

\begin{center}
\begin{tabular}{l}
\begin{alg}{.5\textwidth}{white}
instance Functor [] where
  fmap = map

instance Functor Tree where
  fmap f (Folha a) = f a
  fmap f (Nodo t t') = Nodo (fmap f t) (fmap f t')
\end{alg}
\end{tabular}
\end{center}

\section{Exercícios Resolvidos}

\begin{enumerate}

\item Defina instância de \Functor\ para o construtor de tipo
  \ina{Maybe}.

{\em Solução\/}: 

\begin{center}
\begin{tabular}{l}
\begin{alg}{.5\textwidth}{white}
instance Functor Maybe where
  fmap _ Nothing  = Nothing
  fmap f (Just a) = Just (f a)
\end{alg}
\end{tabular}
\end{center}

\end{enumerate}

%mapM :: (Traversable t, Monad m) => (a -> m b) -> t a -> m (t b)

%Map each element of a structure to a monadic action, evaluate these
%actions from left to right, and collect the results. For a version
%that ignores the results see mapM_.

\section{Mônadas para estruturamento de código}
\label{Essencia-da-PF}

A escrever... baseado em: 

\section{Notação \ina{do}}
\label{Notacao-do}

A notação \ina{do} pode ser usada, em vez de \ina{(>>=)}, para
combinação de valores monádicos. A notação é definida pelas seguintes
equações (que funcionam como regras de tradução), onde \ina{m} é uma
expressão monádica, \ina{cls} é uma sequência não vazia de cláusulas,
componentes da notação, que termina com uma expressão monádica (i.e.~a
última cláusula não é da forma \ina{v <- m}), sendo uma cláusula ou
uma expressão monádica ou uma expressão da forma \ina{v <- m} (onde
\ina{v} é um nome de variável), e \ccls\ a tradução de \cls:

   {\ttfamily
   \begin{center}
   \begin{tabular}{l@{\ $=$\ }l}
      \ddo\ \{ $m$ \}               & $m$ \\
      \ddo\ \{ $m$; \cls\ \}        & $m$ >> \ccls\\
      \ddo\ \{ $v$ <- $m$; \cls\ \} & $m$ >>= $\backslash$ $v$ -> \ccls\\ 
   \end{tabular}
   \end{center}}

Note que, por exemplo, as expressões: 

\begin{center}
\begin{tabular}{l}
\begin{alg}{.5\textwidth}{white}
   do { c <- getChar } 
\end{alg}
\end{tabular}
\end{center}
e
\begin{center}
\begin{tabular}{l}
\begin{alg}{.5\textwidth}{white}
   do { c <- getChar; c' <- getChar} 
\end{alg}
\end{tabular}
\end{center}
não são sintaticamente corretas: a última cláusula em uma notação
\ina{do} tem que ser um valor monádico, i.e.~não pode ser da forma
\ina{c <- m}.

\section{O que é uma mônada} 
\label{Regras-monadicas}

Regras monádicas expressam que tipos monádicos devem ter cláusulas que
podem ser simplificadas como se espera. As simplificações significam
que \ina{return} é um elemento identidade à direita e à esquerda, e
que \ina{>>=} ``se assemelha a um combinator associativo'':

   {\ttfamily
   \begin{center}
   \begin{tabular}{l@{\ $=$\ }l}
       $m$ >>= \return         & $m$\\ 
       \return\ $x$  >>= $f$   & $f\: x$\\
       ($m$ >>= $f$) >>= $g$   & $m$ >>= ($\backslash$ $x$ -> ($f\: x$ >>= $g$))
   \end{tabular}
   \end{center}}

As regras podem ser escritas também, de modo mais simétrico, usando o
seguinte combinador: 

\begin{center}
\begin{tabular}{l}
\begin{alg}{.5\textwidth}{white}
(>=>) :: Monad m => (a -> m b) -> (b -> m c) -> (a -> m c)
(f >=> g) x = f x >>= g
\end{alg}
\end{tabular}
\end{center}

Esse combinador é semelhante ao operador de composição de funções, mas
os tipos dos resultados dos argumentos (funcionais) são monádicos, e a
ordem dos argumentos é inversa à ordem do operador de composição de
funções (i.e.~o argumento é passado para a primeira função, e o
resultado é passado para a segunda função, ao contrário do que ocorre
no caso do operador de composição de funções usual).

Esse combinador é chamado de composição-Kleisli, e está definido na
biblioteca \ina{Control.Monad}, como:

\begin{center}
\begin{tabular}{l}
\begin{alg}{.5\textwidth}{white}
  (f >=> g) x = f x >>= g
\end{alg}
\end{tabular}
\end{center}

As regras monádicas expressas por meio do combinador de
composição-Kleisli indicam que o combinador é associativo com
identidade \ina{return} --- ou seja, é um monóide (um monóide é um
conjunto de valores com uma operação binária associativa e um elemento
identidade). Assim, uma mônada é um monóide se considerarmos o
operador de composição-Kleisli como o operador de composição e
\ina{return} como o elemento identidade. No entanto, o uso de
\ina{>>=} em vez de \ina{>=>} a facilita a programação, e é preferido
como combinador básico da classe \ina{Monad}.

\subsection{Exercícios Resolvidos}
\label{Notacao-do-ex-res}

\begin{enumerate}

\item Expresse as regras monádicas usando a notação \ina{do}.

\begin{center}
\begin{tabular}{l}
\begin{alg}{.5\textwidth}{white}
do {x <- p; return x} = do {p}

do {x <- return e; f x} = do {f e}

do {y <- do {x <- p; f x}; g y} 
= do {x <- p; do {y <- f x; g y}}
= do {x <- p; y <- f x; g y}
\end{alg}
\end{tabular}
\end{center}

\item Suponha que \ina{(>=>)} esteja definido e defina \ina{(>>=)} em
  termos de \ina{(>=>)}.

{\em Solução\/}: 

\begin{center}
\begin{tabular}{l}
\begin{alg}{.5\textwidth}{white}
  (m >>= f) = (id >=> f) m
\end{alg}
\end{tabular}
\end{center}
Alternativamente: 

\begin{center}
\begin{tabular}{l}
\begin{alg}{.5\textwidth}{white}
  (>>=) = flip (id >=>)}.
\end{alg}
\end{tabular}
\end{center}

\end{enumerate}

\subsection{Exercícios}
\label{Notacao-do-ex}

\begin{enumerate}

\item Defina a versão dual da combinador de composição-Kleisli, de
  tipo:

\begin{center}
\begin{tabular}{l}
\begin{alg}{.5\textwidth}{white}
  (<=<):: Monad m => (b->m c) -> (a->m b) -> (a->m c)
\end{alg}
\end{tabular}
\end{center}

Reescreva as regras monádicas usando esse combinador.

\item Mostre que \ina{(f >=> g) . h = (f . h) >=> g}.

\end{enumerate}

\section{Mônada \IO}
\label{Monada-IO}

Incluímos a seguir as operações pré-definidas (definidas em \Prelude)
de entrada e saída de dados.

\begin{itemize}

\item {\bf Funções de leitura do dispositivo de entrada padrão}

  \begin{itemize}

    \item \ina{getChar:: IO Char}: Lê um caractere.

    \item \ina{getLine:: IO String}: Lê uma linha.

    \item \ina{getContents:: IO String}: Retorna todo o conteúdo da
      entrada padrão como uma lista de caracteres, que é lida de fato
      à medida que essa lista é usada.

    \item \ina{interact:: (String -> String) -> IO()}: Recebe uma
      função, digamos $f$, como argumento, e passa toda a entrada como
      uma lista de caracteres para $f$, sendo a cadeia retornada como
      resultado da aplicação de $f$ escrita no dispositivo de saída
      padrão (a entrada é lida de fato à medida que os dados são
      usados por $f$).

  \end{itemize}
    
\item {\bf Funções de escrita no dispositivo de saída padrão}

  \begin{itemize}
    
    \item \ina{putChar:: Char -> IO()}: Escreve o argumento (um caractere).

    \item \ina{putStr:: String -> IO()}: Escreve o argumento (uma
      lista de caracteres).

    \item \ina{putStrLn:: String -> IO()}: Escreve o argumento
      seguido de um caractere de terminação de linha.

    \item \ina{print:: Show a => a -> IO()}: Escreve o resultado de
      converter o argumento em uma lista de caracteres, usando a
      função \ina{show}, e adicionar no final um caractere de
      terminação de linha.

  \end{itemize}

\end{itemize}

Alguns exemplo de programas com entrada e saída de dados
respectivamente nos dispositivos padrões de entrada e saída, são
mostrados a seguir.

\subsection{Entrada e saída em Arquivos}
\label{ES-em-arquivos}

O dispositivo de entrada padrão é denotado em Haskell por \ina{stdin},
e o dispositivo de saída padrão por \ina{stdout}. Uma operação em um
dispositivo de entrada e saída padrão é na maioria das vezes caso
especial de operação mais geral, em que um valor de tipo \ina{Handle},
com informações sobre o dispositivo (usadas pelo sistema de suporte em
tempo de execução da linguagem para gerenciar operações de entrada e
saída de dados), é passado como argumento da operação. Em dispositivos
que não são os dispositivos de entrada e saída padrão, é necessário,
antes de usar as operações de entrada ou saída, abrir o arquivo onde
vai ser realizada a operação. A operação de abrir um arquivo,
\ina{open}, retorna um valor de tipo \ina{Handle}:


\subsection{Exercícios resolvidos}
\label{ex-IO-resolvidos}

\begin{enumerate}

\item Escreva um programa que usa \ina{getChar} para ler dois
  caracteres, um depois do outro, e imprimir uma cadeia contendo os
  dois caracteres lidos, na ordem em que foram lidos, usando
  \ina{putStr}, usando combinador monádico \ina{(>>=)}. Reescreva
  usando a notação \ina{do}.

\begin{center}
\begin{tabular}{l}
\begin{alg}{.5\textwidth}{white}
  main = getChar >>=
            (\ c1 -> getChar >>=
            \ c2 -> getChar -> putStr [c1,c2]))
\end{alg}
\end{tabular}
\end{center}

Usando a notação \ina{do}:

\begin{center}
\begin{tabular}{l}
\begin{alg}{.5\textwidth}{white}
main = do c1 <- getChar
          c2 <- getChar
          putStr [c1,c2]
\end{alg}
\end{tabular}
\end{center}

\item Considere as duas definições abaixo:

\begin{center}
\begin{tabular}{l}
\begin{alg}{.5\textwidth}{white}
copiaAteLinhaVazia = do lin <- getLine
                        let copia = do
                              if (lin == "")
                                then return ()
                                else do putStrLn lin
                                        lin <- getLine
                                        copia
                        copia
\end{alg}
\end{tabular}
\end{center}

\begin{center}
\begin{tabular}{l}
\begin{alg}{.5\textwidth}{white}
copiaAteLinhaVazia = do 
  l <- getLine
  if null l then return ()
            else do {putStrLn l; copiaAteLinhaVazia}
\end{alg}
\end{tabular}
\end{center}

Uma delas não copia linhas da entrada até que seja lida uma linha
vazia. Qual e porquê?

{\em Solução\/}: A primeira. 

A primeira função lê uma linha e só termina se esta linha for uma
linha vazia; caso contrário, qualquer chamada a
\ina{copiaAteLinhaVazia} não termina: a cláusula \ina{lin <- getLine}
usada após o teste de \ina{(lin == "")}, na notação \ina{do}, não
modifica o valor denotado pela variável \ina{lin} usada nesse
teste. Essa cláusula cria nova variável, de nome \ina{lin}, que tem o
mesmo nome de variável que já existe (em escopo mais externo; a
variável mais externa fica invisível devido à criação da variável com
mesmo nome em escopo mais interno). Podemos reescrever a primeira
função \ina{copiaAteLinhaVazia} sem usar a notação \ina{do} para
tornar isso explícito:

\begin{center}
\begin{tabular}{l}
\begin{alg}{.5\textwidth}{white}
copiaAteLinhaVazia = 
  getLine >>= \ lin -> let copia = if (lin == "")
                                     then return ()
                                     else putStrLn lin >> 
                                          getLine >>= \ lin -> 
                                          copia
                        in copia
\end{alg}
\end{tabular}
\end{center}

\item Escreva uma programa \ina{wc} que se comporte como o programa
  \ina{wc} existente no Linux.  Ou seja, o programa tem o nome de um
  arquivo-texto como argumento e imprime o número de linhas, palavras
  e caracteres existentes no arquivo.

{\em Solução\/}: 

A solução a seguir usa a função \ina{words}, definida em
\ina{Prelude}, para calcular o número de palavras do texto.

\begin{center}
\begin{tabular}{l}
\begin{alg}{.5\textwidth}{white}
wc :: String -> (Int,Int,Int)
wc s = (length s, words s, filter (=='\n') s)

head':: [String] -> String
head' [] = error "Nome de arquivo esperado"
head' s  = head s

main = getArgs >>= print . wc . head' 
\end{alg}
\end{tabular}
\end{center}

A solução a seguir não usa a função \ina{words}; a função \ina{wc'}
percorre a entrada apenas uma vez calculando o número de caracteres,
palavras e linhas (e ilustra como calcular o número de palavras por
meio de um booleano que indica se o caractere corrente é componente de
uma palavra ou não):

\begin{center}
\begin{tabular}{l}
\begin{alg}{.5\textwidth}{white}
import Data.Char (isSpace)

wc :: String -> (Int,Int,Int)
wc = wc' False

wc' _ []                        = (0,0,0)
wc' inWord (c:r) 
  | c_isSpace                   = (numChars+1, numWords', numLins')
  | otherwise                   = (numChars+1, numWords, numLins)
  where 
    c_isSpace                   = isSpace c
    (numChars,numWords,numLins) = wc' (not c_isSpace) r
    numWords'
      | inWord                  = numWords+1 
      | otherwise               = numWords
    numLins'
      | c == '\n'               = numLins+1
      | otherwise               = numLins

head':: [String] -> String
head' [] = error "Nome de arquivo esperado"
head' s  = head s

main = getArgs >>= print . wc . head' 
\end{alg}
\end{tabular}
\end{center}

\item \label{Ex:uso-ST} Escreva uma definição de uma função que receba
  um inteiro positivo \ina{n} e retorne o \ina{n}-ésimo número de
  Fibonacci, usando a mônada \ina{ST} para evitar chamadas recursivas
  de funções. 

\begin{center}
\begin{tabular}{l}
\begin{alg}{.5\textwidth}{white}
fib :: Int -> ST s Integer
fib n = do a <- newSTRef 0
           b <- newSTRef 1
           for_ n (do x <- readSTRef a
                     y <- readSTRef b
                     writeSTRef a y
                     writeSTRef b (x+y)
                 )
           readSTRef a
\end{alg}
\end{tabular}
\end{center}

%A função \ina{($!)} é usada para forçar a avaliação da soma
%(\ina{x+y}); ($!) é definida em \ina{Prelude} como:
%
%\begin{center}
%\begin{tabular}{l}
%\begin{alg}{.5\textwidth}{white}
%($!) :: (a->b) -> a -> b
%f $! x = x `seq` f x
%\end{alg}
%\end{tabular}
%\end{center}

%A função \ina{seq} força a avaliação do argumento (\ina{x}) antes de
%fornecer o resultado da avaliação da aplicação da função (\ina{f}) a
%este argumento. O uso da função \ina{seq} (e portanto \ina{S!}) deve
%ser feito somente para tornar o desempenho mais eficiente (quando
%necessário).

\end{enumerate}

\subsection{Exercícios}
\label{ex-IO}

\begin{enumerate}

\item ... ... 

\end{enumerate}

\section{Mônada \Maybe}
\label{Monada-Maybe}

... ...

\section{Referências na mônada \IO}

O módulo \ina{Data.IORef} contém definições para suporte ao uso de
referências (variáveis no sentido de programação imperativa, i.e.~que
armazenam um valor que pode ser modificado durante a execução de um
programa) em Haskell, na mônada \ina{IO}.

As funções definidas no módulo \ina{Data.IORef} são:

\begin{itemize}
\item \ina{newIORef :: a -> IO (IORef a)}: Recebe um valor e retorna
  uma referência para esse valor.
\item \ina{readIORef :: IORef a -> IO a}: Recebe uma referência e
  retorna o valor apontado pela referência.
\item \ina{writeIORef :: IORef a -> a -> IO()}: Recebe uma
  referência e um valor e retorna uma ação que armazena o valor na
  referência.
\item \ina{modifyIORef :: IORef a -> (a->a) -> IO()}:
  Recebe uma referência e uma função $f$ e retorna uma ação que
  armazena o valor resultante de aplicar $f$ ao valor corrente
  armazenado na referência. Ou seja, modifica o valor $v$ armazenado
  na referência pelo valor dado por $f$ $v$.
\end{itemize}

{\tt \begin{tabbing}
{\em while\/}:: {\em IO Bool\/} -> {\em IO\/}() -> {\em IO\/}()\\
{\em while test action\/} = \\
xx\=\kill
  \> do \=$b$ <- {\em test}\\
  \>    \>if $b$ \=then do \={\em action}\\\
  \>    \>       \>        \>{\em while test action}\\
  \>    \>       \>else return ()
\end{tabbing}}

{\tt readRevWrite = getLine >>= apply rev >>= putLine}

\begin{tabular}[t]{ll} M\^onada-Erro & 
  {\small {\tt \begin{tabular}[t]{l}
    data Err t = Ok t | Error\\
    return a = Ok a\\
    m >>= f = case m of\\
      \ \ \ (Ok a) -> f a \\
      \ \ \ Error -> Error
  \end{tabular} } } 
\end{tabular}   

Para manipular valores sequencialmente, podemos usar combinadores mais
simples que os combinadores monádicos, tipicamente \fmap, da classe
\Functor\ (seção \ref{Funtores}), ou combinadores da classe
\Applicative\ (seção \ref{Applicative}), abordadas a seguir.

%\section{Mônada \ina{ST}}
%\label{Monada-ST}



	

\pagestyle{fancyplain}

\bibliographystyle{plain}
\bibliography{livro}

\printindex

\end{document}
