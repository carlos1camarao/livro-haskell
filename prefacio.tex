\chapter{Prefácio}

Este é um livro sobre introdução a programação funcional em Haskell.
O livro é destinado a estudantes de graduação de cursos universitários
de Ciência da Computação e cursos correlatos. O objetivo do livro é
apresentar as ideias principais da programação funcional, como ela se
desenvolveu e vem se desenvolvendo a partir das ideias de programação
com tipos de dados polimórficos, tipos algébricos recursivos,
sobrecarga dependente de contexto e programação monádica.

Resolvemos escrever este livro por considerar que os livros atuais não
abordam ou enfocam os temas acima como achamos que deveria ser
feito. Em particular, consideramos que a definição e o uso de tipos
polimórficos, em particular de tipos de dados algébricos recursivos, é
essencial para facilitar na tarefa de programação, para definição de
funções que manipulam tais estruturas de dados, tipicamente listas e
árvores.

Na presença de recursos para definição de tipos polimórficos e de
tipos definidos recursivamente, o fato da linguagem ser estaticamente
tipada facilita enormemente a tarefa de programação. É essencial
apontar que é a possibilidade de definir tipos/estruturas de dados
recursivamente que é a característica mais importante, em vez da
possibilidade de definir funções recursivamente. É essa ideia básica
que motiva a primeira parte deste livro.

Procuramos também salientar a importância do mecanismo que permite a
definição e uso de funções sobrecarregadas para manipulação dessas
estruturas de dados polimórficas, como tipicamente as funções filter,
map e principalmente fold, mas também muitas outras.

A segunda parte do livro aborda a programação com funtores, funtores
aplicativos, e a programação monádica. ...

A terceira parte do livro aborda o tema de correção de programas. ...





