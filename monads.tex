\chapter{Programação com Ações e Programação Monádica}
\label{Monadas}

\section{Introdução}

A solução adotada por \Haskell\ para permitir o uso de ações com
efeito colateral é explicada a seguir. Chamamos de {\em ação\/} uma
construção da linguagem cuja execução provoca uma ``mudança de
estado'', como modificar o valor armazenado em uma variável ou
realizar alguma operação de entrada e saída.

Observação: {\em Ação com efeito colateral\/} é um nome mais ortodoxo,
uma vez que a palavra {\em ação\/} é usada em Haskell para denotar
qualquer valor monádico --- isto é, qualquer valor de tipo (monádico)
$m$ $a$, para qualquer $a$ e qualquer $m$ definido como instância da
classe \ina{Monad}. No entanto, o uso do termo ``ação'' para um valor
desse tipo coloca uma qualificação indevida em valores monádicos: se
não provocam efeito colateral, esses valores são valores como outros
quaisquer. Portanto, usamos consistentemente ``ação'' com o sentido
que é usualmente em computação o sentido de ``ação com efeito
colateral''.

A solução adotada em Haskell é baseada na seguinte ideia básica:

  \begin{quotation}
     {\bf Um construtor é usado para distinguir uma ação do resultado
       que ela produz.}
  \end{quotation}

Em Haskell, valores de tipo {\tt \IO\ $a$}, para um tipo \ina{a}
qualquer, s\~ao a\c{c}\~oes (com efeito colateral). O tipo que é
instância de \ina{a} é o tipo do valor retornado pela execução da
ação.

Valores do tipo \ina{ST s a} também são ações. Neste tipo, a variável
de tipo \ina{a} pode ser instanciada, como qualquer variável de tipo,
mas a variável de tipo \ina{s} tem uma característica especial: ela
não pode ser instanciada em argumentos de \ina{runST} (veja explicação
a seguir).

Nesses argumentos, a variável \ina{s} existe apenas para identificar
ou dar um nome a, digamos, um ``fluxo de execução''. O termo ``fluxo
de execução'' é apenas um nome dado ao argumento de \ina{runST} (uma
sequência de ações quaisquer), de tipo \ina{ST s a} (para algum
\ina{a} e, como vamos ver, para um \ina{s} que não pode ser
instanciado). Tipos (\ina{STRef s a}) de variáveis que podem ser
modificadas são ``etiquetadas com esse \ina{s}''. Isso é feito para
que ações que modificam o valor dessas variáveis só possam existir
nesse argumento (diz-se ``nesse fluxo de execução''). Assim, uma
variável etiquetada não pode ser usada novamente em outro fluxo de
execução. No entanto, o resultado obtido na execução desse argumento
(nesse ``fluxo de execução'') deve poder ser usado em outro fluxo de
execução.

É possível em Haskell, dessa forma, modificar o valor armazenado em
uma ``variável'', no sentido de ``variável'' em linguagens e programas
imperativos. Em Haskell, tais entidades são chamadas de referências e
devem ter tipo \ina{STRef s a}, para algum \ina{a} e para algum
\ina{s}. (O tipo \ina{IORef} também pode ser usado para modificação do
valor armazenado em variáveis, mas não carrega etiqueta, e por isso o
valor armazenado não pode ``sair'' da mônada \ina{IO}). O tipo
\ina{STRef s a} indica que a referência (variável, no sentido
imperativo) é local ao argumento de \ina{runST} (diz-se ``é local a um
determinado 'fluxo de execução'); essa etiqueta não pode ``sair'' da
mônada \ina{ST}, ou seja, não pode ser usada em nenhum outro argumento
a \ina{runST}.  As operações abaixo, que manipulam valores de tipo
\ina{STRef s a}, têm resultado de tipo \ina{ST s a} (para algum
\ina{a},\ina{s}):

\begin{center}
\begin{tabular}{l}
\begin{alg}{.5\textwidth}{white}
newSTRef :: a -> ST s (STRef s a)
readSTRef :: STRef s a -> ST s a
writeSTRef :: STRef s a -> a -> ST s ()
\end{alg}
\end{tabular}
\end{center}

A função \ina{newSTRef} recebe um valor de um tipo $a$ e retorna uma
ação que, quando executada, retorna uma referência para esse valor.

A função \ina{readSTRef} recebe uma referência a um valor de tipo
\ina{a} e retorna uma ação que, quando executada, retorna o valor
armazenado no endereço denotado por essa referência.

A função \ina{writeSTRef} recebe uma referência a um valor de tipo \ina{a}
e um valor de tipo \ina{a} e retorna uma ação que, quando executada,
armazena esse valor no endereço denotado pela referência.

Agora você poderá entender precisamente o que temos dito. O único modo
de usar um valor produzido pela execução de uma ação de tipo \ina{ST s a} 
(para algum \ina{a}) em um contexto que não envolve o construtor
\ina{ST}, é por meio do uso da função \ina{runST}, que tem um tipo
peculiar:

\prog{\ina{(forall s. ST s a) -> a}}

Esse é um tipo diferente dos tipos que vimos até agora. É chamado de
um tipo polimórfico de {\em grau 2\/} (em inglês, {\em rank 2\/}). Em
todos os tipos que vimos até agora, o quantificador \ina{forall}
ocorria externamente, antes do tipo não quantificado. Tais tipos têm
{\em grau 1\/}. Um tipo de grau 2, como o de \ina{runST}, requer que o
argumento tenha um tipo no qual o estado (\ina{s}) não é instanciado.

Considere por exemplo a diferença entre 
  $t_1 = $ \ina{forall a. a->Int} e 
  $t_2 = $ \ina{(forall a. a)->Int}. 
O único valor de tipo \ina{forall a. a} é o que denota não-terminação
($\perp$). No entanto, a variável de tipo \ina{a} pode ser instanciada
para fornecer várias funções com tipos que são instâncias de $t_2$
(por exemplo, o tipo de \ina{Int -> Int} é instância de $t_2$).

O tipo de \ina{runST} evita o uso de \ina{runST} em expressões como: 

\begin{center}
\begin{tabular}{l}
\begin{alg}{.5\textwidth}{white}
let t = runST (newSTRef True)
    f = runST (writeSTRef t False)
in runST (readSTRef t)
\end{alg}
\end{tabular}
\end{center}

Essa expressão não é tipável: note que \ina{newSTRef True} tem tipo
\ina{ST s (STRef s Bool)}. A expressão \ina{runST (newSTRef Bool)}
teria então que ter tipo \ina{STRef s Bool}, ou seja, \ina{a} teria
que ser unificado com \ina{STRef s Bool}. No entanto, isso modificaria
o tipo \ina{\ST s a}, requerido por \ina{runST}, porque o tipo do
resultado de \ina{runST (STRef s Bool)} passa a depender de \ina{s} (é
comum dizer: a variável de tipo $s$ não pode ``escapar'' para o tipo
do resultado).

A presença da variável \ina{s} em um valor de tipo \ina{STRef s t}
(para um \ina{t} qualquer) explica porque existem dois tipos distintos
de referências em Haskell, construídas com \ina{STRef} e com
\ina{IORef}. O construtor \ina{IORef} não tem tal variável de estado;
um valor é construído apenas com um valor do tipo que desejado (para
qualquer \ina{t}, o tipo \ina{IORef t} é um tipo de referência para
valor de tipo \ina{t}). Em vez de existir uma condição de uso de
\ina{s} (que requer que \ina{s} não possa ser instanciada e não possa,
assim, ser usada em tipo de outro argumento para \ina{runST},
requer-se que não haja, ou pelo menos não seja usada em geral, nenhuma
função que ``saia da mônada'' \ina{IO} (ou seja, requer-se que não
haja ou não seja usada em geral nenhuma função como
\ina{unsafePerformIO}).

O exercício resolvido \ref{Ex:uso-ST} ilustra o uso da mônada
\ina{ST}.

Para ações de entrada e saída de dados, é usado o construtor \ina{IO};
por exemplo, a ação de ler um caractere, \ina{getChar}, tem tipo
\ina{IO Char}. Como já mencionado, o uso do construtor \ina{IO} indica
que se trata de uma ação, e distingue a ação do resultado obtido pela
execução da ação, que é um valor de tipo \ina{Char}.

Uma ação de entrada e saída pode ser executada, para se obter o valor
resultante de sua execução, mas:

  \begin{quotation}
    {\bf Não é possível definir função que recebe ação de tipo \ina{IO t}, 
         para um \ina{t} qualquer, e retorna valor de tipo \ina{t}.}
  \end{quotation}

Não se pode definir, por exemplo, uma função de tipo {\tt
  \IO\ \Char\ -> \Char}.  Note no entanto que existe uma função,
disponível na versão de Haskell implementada pelo compilador GHC
\cite{GHC}, que tem tipo \ina{IO a -> a}: a função
\ina{unsafePerformIO}.  Mas, como o próprio nome indica, seu uso não é
em geral seguro, ela só deve ser usada em circunstâncias especiais, em
que a mudança de estado provocada pela ação de entrada ou saída não
tem efeito indesejado. O uso de uma função como \ina{unsafePerformIO}
pode destruir a relevância da distinção entre a ação e o valor
retornado em sua execução.

As condições sobre o uso de ações em Haskell são essenciais para
garantir que a linguagem seja uma linguagem funcional pura, com {\em
  transparência referencial}.

% Essas condições, chamadas de CLFP ({\em condições para linguagem
%  funcional ser pura\/}) são: i) distinção entre o tipo de uma ação e
%  o resultado de sua execução --- em Haskell, essa distinção é feita
%  simplesmente com introdução de um construtor no tipo da ação ---, e
%  ii) não existência de função que recebe ação e retorna valor que
%  não envolve ação. ...

Em Haskell, a função \ina{main}, que representa o efeito de executar o
programa resultante de uma compilação, é uma ação, de tipo \ina{IO()}
(o parâmetro \ina{()} do construtor \ina{IO} indica que não há valor
útil retornado como resultado da execução dessa ação). A execução de
toda ação em um programa Haskell é iniciada a partir da execução da
ação definida na função \ina{main}.

%O fato de ações serem manipuladas em Haskell por meio do uso de
%funções monádicas (\return\ e {\tt >>=}) é uma questão secundária,
%embora importante. A questão principal, mais básica, é a explicitada
%pela CLFP acima.

%As únicas formas de manipular ações em Haskell são: i) executar uma
%ação, possivelmente obtendo o resultado dessa execução, e ii) compor
%uma ação $a_1$ sequencialmente com outra ação $a_2$ (a execução da
%ação composta consiste em executar $a_1$ e em seguida $a_2$). A
%execução da ação $a_2$ pode usar o resultado retornado pela execução
%de $a_1$.

Existem ações básicas de leitura e escrita, como por exemplo
\ina{getChar} e \ina{putChar} para leitura e escrita de um caractere,
e \ina{getLine} e \ina{putStr} para leitura e escrita de uma lista de
caracteres, que serão abordadas na seção \ref{Monada-IO}.

Na seção seguinte é apresentada uma introdução a mônadas em Haskell.
Em Haskell, ações e valores monádicos em geral são usados como
argumentos de ``combinadores'' (funções de ordem superior, não locais
a outras funções, geralmente usadas na construção e combinação de
valores de um determinado tipo). Funções de ordem superior proveem uma
maneira de estruturar programas, abordada na seção seguinte.

\section{Mônadas e estruturação de programas}
\label{Monadas-e-estruturacao-de-programas}

Existem dois {\em combinadores monádicos\/} principais, \ina{return} e
\ina{>>=}. É comum dizer: \ina{return} faz com que se entre em uma
mônada, e \ina{>>=} (costuma-se dizer, em inglês, {\em bind\/})
permite sequenciar ações monádicas. Esses combinadores constituem a
ferramenta principal usada na manipulação de ações e de valores
monádicos em geral em Haskell.

%Para mônadas que são ações de entrada e saída, não há como ``sair da
%mônada'', isto é, obter um valor não monádico --- mais precisamente,
%não é possível obter valor de algum tipo $t$ a partir de um valor
%monádico \IO\ $t$ (no entanto, já falamos anteriormente de
%\unsafePerformIO).

%Para mônadas que não são ações, no entanto, é possível ``sair da
%mônada'', isto é, simplesmente avaliar a expressão monádica para obter
%o valor resultante, não monádico.

Em Haskell, uma mônada é uma classe definida como a seguir:

\begin{center}
\begin{tabular}{l}
\begin{alg}{.5\textwidth}{white}
class Monad m where
  (>>=) :: m a -> (a -> m b) -> m b
  (>>) :: m a -> m b -> m b
  return :: a -> m a
  fail :: String -> m a
\end{alg}
\end{tabular}
\end{center}

Dado um tipo monádico (instância da classe \ina{Monad}) \ina{m t},
vamos chamar de \ina{t} o tipo-alvo do tipo monádico. Valor monádico é
um valor de tipo monádico e expressão monádica é uma expressão de tipo
monádico.

Em qualquer instância da classe \ina{Monad}, a função \ina{(>>=)}
compõe sequencialmemte dois valores monádicos, sendo a combinação
feita de modo que o valor resultante da avaliação (ou execução, no
caso de ações) do primeiro valor monádico é passado como argumento do
segundo.

A função \ina{(>>)} é semelhante: compõe sequencialmente dois valores
monádicos, mas a combinação é feita de modo que o valor resultante da
avaliação (ou execução, no caso de ações) do primeiro valor monádico é
descartado (em vez de ser passado para o segundo). A função \ina{(>>)}
pode ser facilmente definida em termos de \ina{(>>=)}:

\begin{center}
\begin{tabular}{l}
\begin{alg}{.5\textwidth}{white}
(m >> m') = m >>= \ _ -> m'
\end{alg}
\end{tabular}
\end{center}

A função \ina{return} apenas transforma o argumento em um valor
monádico, em geral simplesmente etiquetando o argumento com um
construtor do tipo-alvo do tipo monádico.

A função \ina{fail} receve uma cadeia de caracteres e retorna um valor
de tipo monádico que indica uma situação de erro. Essa função não é
parte da definição matemática de mônada, é usada com a notação
\ddo\ (veja seção \ref{Notacao-do} a seguir), quando ocorre falha em
casamento de padrões.

\subsection{Exercícios Resolvidos}

\begin{enumerate}

\item Escreva uma função \ina{sequence_} que recebe uma lista de ações
  e retorne uma ação que consiste em executá-las sequencialmente, uma
  após a outra, desprezando o resultado obtido com a sua execução. 

{\em Solução\/}: 

\begin{center}
\begin{tabular}{l}
\begin{alg}{.5\textwidth}{white}
sequence_ :: Monad m => [m a] -> m ()
sequence_ = foldr (>>) (return ())
\end{alg}
\end{tabular}
\end{center}

O uso de \ina{_} como sufixo do nome de uma função é usado em geral em
Haskell para indicar que os resutado da ação é \ina{()} (sendo
desconsiderados eventuais resultados de argumentos que são ações).

\item Escreva uma função \ina{for_} que recebe um inteiro $n$ e uma
  ação $m$ e retorna uma ação que consiste em executar $m$ exatamente
  $n$ vezes, desconsiderando os resultados.

{\em Solução\/}: 

\begin{center}
\begin{tabular}{l}
\begin{alg}{.5\textwidth}{white}
for_ :: Monad m => Int -> m a -> m ()
for_ n = sequence_  . replicate n
\end{alg}
\end{tabular}
\end{center}

\item Escreva uma função \ina{sequence} que recebe uma lista de ações
  e retorne uma ação que consiste em executá-las sequencialmente, uma
  após a outra, retornando a lista dos resultados obtidos com a sua
  execução.

\begin{center}
\begin{tabular}{l}
\begin{alg}{.5\textwidth}{white}
sequence :: Monad m => [m a] -> m [a]
sequence = foldr consa (return [])
  where consa ma m = do {a <- ma; x <- m; return (a:x)}
\end{alg}
\end{tabular}
\end{center}

\item Escreva função \ina{mapM_} que recebe uma lista de funções, cada
  uma delas retornando uma ação, e retorne uma ação que consiste em
  executar as ações resultantes de aplicar á-las sequencialmente, uma
  após a outra, retornando a lista dos resultados obtidos com a sua
  execução.

\begin{center}
\begin{tabular}{l}
\begin{alg}{.5\textwidth}{white}
sequence :: Monad m => [m a] -> m [a]
sequence = foldr consa (return [])
  where consa ma m = do {a <- ma; x <- m; return (a:x)}
\end{alg}
\end{tabular}
\end{center}
The two new map functions are:

\begin{center}
\begin{tabular}{l}
\begin{alg}{.5\textwidth}{white}
mapM_ f = sequence_ . map f
mapM f = sequence . map f

mapM :: (Traversable t, Monad m) => (a -> m b) -> t a -> m (t b)
\end{alg}
\end{tabular}
\end{center}

%Map each element of a structure to a monadic action, evaluate these
%actions from left to right, and collect the results. For a version
%that ignores the results see mapM_.

\begin{center}
\begin{tabular}{l}
\begin{alg}{.5\textwidth}{white}
mapM_ :: (Foldable t, Monad m) => (a -> m b) -> t a -> m ()
\end{alg}
\end{tabular}
\end{center}

\end{enumerate}

\section{Classes \ina{Foldable}, \ina{Traversable}}
\label{Foldable-Traversable}

\section{Classes \ina{Functor}, \ina{Applicative}}
\label{Applicative}

\section{Funtores}
\label{Funtores}

Funtores em Haskell são construtores de tipos que são instâncias da
classe \ina{Functor}, definida no \ina{Prelude}, para os quais é
definida uma função, chamada em Haskell de $\fmap$:

\begin{center}
\begin{tabular}{l}
\begin{alg}{.5\textwidth}{white}
class Functor f where
  fmap :: (a -> b) -> f a -> f b
\end{alg}
\end{tabular}
\end{center}

Para cada construtor de tipos \ina{f}, \ina{fmap g} aplica a função
\ina{g} a todos os elementos da estrutura de dados construída por meio
de \ina{f}. Note que \ina{f} é um construtor de tipos.
% (veja seção \ref{Kinds}).

Por exemplo, as funções \ina{map} e \ina{mapTree} com os tipos a
seguir devem ser definidas de modo a respectivamente aplicar uma
função fornecida como argumento a cada um dos elementos de uma lista
ou árvore:

\begin{center}
\begin{tabular}{l}
\begin{alg}{.5\textwidth}{white}
map :: (a -> b) -> [a] -> [b]
mapTree :: (a -> b) -> Tree a -> Tree b
\end{alg}
\end{tabular}
\end{center}

A função \ina{map} é uma instância de \ina{fmap} para listas (i.e.~uma
instância na qual \ina{f} é o construtor de tipos \ina{[]}) e,
analogamente, \ina{mapTree} é uma instância de \ina{fmap} para árvores
(i.e.~uma instância na qual \ina{f} é igual ao construtor de tipos
\ina{Tree}).  O tipo \ina{Tree} deve, é claro, ser visível para que
\ina{Tree} possa ser usado; a definição pode ser como a seguir:

\begin{center}
\begin{tabular}{l}
\begin{alg}{.5\textwidth}{white}
data Tree a = Folha a | Nodo (Tree a) (Tree a)
\end{alg}
\end{tabular}
\end{center}

As definições podem ser feitas como a seguir:

\begin{center}
\begin{tabular}{l}
\begin{alg}{.5\textwidth}{white}
instance Functor [] where
  fmap = map

instance Functor Tree where
  fmap f (Folha a) = f a
  fmap f (Nodo t t') = Nodo (fmap f t) (fmap f t')
\end{alg}
\end{tabular}
\end{center}

\section{Exercícios Resolvidos}

\begin{enumerate}

\item Defina instância de \Functor\ para o construtor de tipo
  \ina{Maybe}.

{\em Solução\/}: 

\begin{center}
\begin{tabular}{l}
\begin{alg}{.5\textwidth}{white}
instance Functor Maybe where
  fmap _ Nothing  = Nothing
  fmap f (Just a) = Just (f a)
\end{alg}
\end{tabular}
\end{center}

\end{enumerate}

%mapM :: (Traversable t, Monad m) => (a -> m b) -> t a -> m (t b)

%Map each element of a structure to a monadic action, evaluate these
%actions from left to right, and collect the results. For a version
%that ignores the results see mapM_.

\section{Mônadas para estruturamento de código}
\label{Essencia-da-PF}

A escrever... baseado em: 

\section{Notação \ina{do}}
\label{Notacao-do}

A notação \ina{do} pode ser usada, em vez de \ina{(>>=)}, para
combinação de valores monádicos. A notação é definida pelas seguintes
equações (que funcionam como regras de tradução), onde \ina{m} é uma
expressão monádica, \ina{cls} é uma sequência não vazia de cláusulas,
componentes da notação, que termina com uma expressão monádica (i.e.~a
última cláusula não é da forma \ina{v <- m}), sendo uma cláusula ou
uma expressão monádica ou uma expressão da forma \ina{v <- m} (onde
\ina{v} é um nome de variável), e \ccls\ a tradução de \cls:

   {\ttfamily
   \begin{center}
   \begin{tabular}{l@{\ $=$\ }l}
      \ddo\ \{ $m$ \}               & $m$ \\
      \ddo\ \{ $m$; \cls\ \}        & $m$ >> \ccls\\
      \ddo\ \{ $v$ <- $m$; \cls\ \} & $m$ >>= $\backslash$ $v$ -> \ccls\\ 
   \end{tabular}
   \end{center}}

Note que, por exemplo, as expressões: 

\begin{center}
\begin{tabular}{l}
\begin{alg}{.5\textwidth}{white}
   do { c <- getChar } 
\end{alg}
\end{tabular}
\end{center}
e
\begin{center}
\begin{tabular}{l}
\begin{alg}{.5\textwidth}{white}
   do { c <- getChar; c' <- getChar} 
\end{alg}
\end{tabular}
\end{center}
não são sintaticamente corretas: a última cláusula em uma notação
\ina{do} tem que ser um valor monádico, i.e.~não pode ser da forma
\ina{c <- m}.

\section{O que é uma mônada} 
\label{Regras-monadicas}

Regras monádicas expressam que tipos monádicos devem ter cláusulas que
podem ser simplificadas como se espera. As simplificações significam
que \ina{return} é um elemento identidade à direita e à esquerda, e
que \ina{>>=} ``se assemelha a um combinator associativo'':

   {\ttfamily
   \begin{center}
   \begin{tabular}{l@{\ $=$\ }l}
       $m$ >>= \return         & $m$\\ 
       \return\ $x$  >>= $f$   & $f\: x$\\
       ($m$ >>= $f$) >>= $g$   & $m$ >>= ($\backslash$ $x$ -> ($f\: x$ >>= $g$))
   \end{tabular}
   \end{center}}

As regras podem ser escritas também, de modo mais simétrico, usando o
seguinte combinador: 

\begin{center}
\begin{tabular}{l}
\begin{alg}{.5\textwidth}{white}
(>=>) :: Monad m => (a -> m b) -> (b -> m c) -> (a -> m c)
(f >=> g) x = f x >>= g
\end{alg}
\end{tabular}
\end{center}

Esse combinador é semelhante ao operador de composição de funções, mas
os tipos dos resultados dos argumentos (funcionais) são monádicos, e a
ordem dos argumentos é inversa à ordem do operador de composição de
funções (i.e.~o argumento é passado para a primeira função, e o
resultado é passado para a segunda função, ao contrário do que ocorre
no caso do operador de composição de funções usual).

Esse combinador é chamado de composição-Kleisli, e está definido na
biblioteca \ina{Control.Monad}, como:

\begin{center}
\begin{tabular}{l}
\begin{alg}{.5\textwidth}{white}
  (f >=> g) x = f x >>= g
\end{alg}
\end{tabular}
\end{center}

As regras monádicas expressas por meio do combinador de
composição-Kleisli indicam que o combinador é associativo com
identidade \ina{return} --- ou seja, é um monóide (um monóide é um
conjunto de valores com uma operação binária associativa e um elemento
identidade). Assim, uma mônada é um monóide se considerarmos o
operador de composição-Kleisli como o operador de composição e
\ina{return} como o elemento identidade. No entanto, o uso de
\ina{>>=} em vez de \ina{>=>} a facilita a programação, e é preferido
como combinador básico da classe \ina{Monad}.

\subsection{Exercícios Resolvidos}
\label{Notacao-do-ex-res}

\begin{enumerate}

\item Expresse as regras monádicas usando a notação \ina{do}.

\begin{center}
\begin{tabular}{l}
\begin{alg}{.5\textwidth}{white}
do {x <- p; return x} = do {p}

do {x <- return e; f x} = do {f e}

do {y <- do {x <- p; f x}; g y} 
= do {x <- p; do {y <- f x; g y}}
= do {x <- p; y <- f x; g y}
\end{alg}
\end{tabular}
\end{center}

\item Suponha que \ina{(>=>)} esteja definido e defina \ina{(>>=)} em
  termos de \ina{(>=>)}.

{\em Solução\/}: 

\begin{center}
\begin{tabular}{l}
\begin{alg}{.5\textwidth}{white}
  (m >>= f) = (id >=> f) m
\end{alg}
\end{tabular}
\end{center}
Alternativamente: 

\begin{center}
\begin{tabular}{l}
\begin{alg}{.5\textwidth}{white}
  (>>=) = flip (id >=>)}.
\end{alg}
\end{tabular}
\end{center}

\end{enumerate}

\subsection{Exercícios}
\label{Notacao-do-ex}

\begin{enumerate}

\item Defina a versão dual da combinador de composição-Kleisli, de
  tipo:

\begin{center}
\begin{tabular}{l}
\begin{alg}{.5\textwidth}{white}
  (<=<):: Monad m => (b->m c) -> (a->m b) -> (a->m c)
\end{alg}
\end{tabular}
\end{center}

Reescreva as regras monádicas usando esse combinador.

\item Mostre que \ina{(f >=> g) . h = (f . h) >=> g}.

\end{enumerate}

\section{Mônada \IO}
\label{Monada-IO}

Incluímos a seguir as operações pré-definidas (definidas em \Prelude)
de entrada e saída de dados.

\begin{itemize}

\item {\bf Funções de leitura do dispositivo de entrada padrão}

  \begin{itemize}

    \item \ina{getChar:: IO Char}: Lê um caractere.

    \item \ina{getLine:: IO String}: Lê uma linha.

    \item \ina{getContents:: IO String}: Retorna todo o conteúdo da
      entrada padrão como uma lista de caracteres, que é lida de fato
      à medida que essa lista é usada.

    \item \ina{interact:: (String -> String) -> IO()}: Recebe uma
      função, digamos $f$, como argumento, e passa toda a entrada como
      uma lista de caracteres para $f$, sendo a cadeia retornada como
      resultado da aplicação de $f$ escrita no dispositivo de saída
      padrão (a entrada é lida de fato à medida que os dados são
      usados por $f$).

  \end{itemize}
    
\item {\bf Funções de escrita no dispositivo de saída padrão}

  \begin{itemize}
    
    \item \ina{putChar:: Char -> IO()}: Escreve o argumento (um caractere).

    \item \ina{putStr:: String -> IO()}: Escreve o argumento (uma
      lista de caracteres).

    \item \ina{putStrLn:: String -> IO()}: Escreve o argumento
      seguido de um caractere de terminação de linha.

    \item \ina{print:: Show a => a -> IO()}: Escreve o resultado de
      converter o argumento em uma lista de caracteres, usando a
      função \ina{show}, e adicionar no final um caractere de
      terminação de linha.

  \end{itemize}

\end{itemize}

Alguns exemplo de programas com entrada e saída de dados
respectivamente nos dispositivos padrões de entrada e saída, são
mostrados a seguir.

\subsection{Entrada e saída em Arquivos}
\label{ES-em-arquivos}

O dispositivo de entrada padrão é denotado em Haskell por \ina{stdin},
e o dispositivo de saída padrão por \ina{stdout}. Uma operação em um
dispositivo de entrada e saída padrão é na maioria das vezes caso
especial de operação mais geral, em que um valor de tipo \ina{Handle},
com informações sobre o dispositivo (usadas pelo sistema de suporte em
tempo de execução da linguagem para gerenciar operações de entrada e
saída de dados), é passado como argumento da operação. Em dispositivos
que não são os dispositivos de entrada e saída padrão, é necessário,
antes de usar as operações de entrada ou saída, abrir o arquivo onde
vai ser realizada a operação. A operação de abrir um arquivo,
\ina{open}, retorna um valor de tipo \ina{Handle}:


\subsection{Exercícios resolvidos}
\label{ex-IO-resolvidos}

\begin{enumerate}

\item Escreva um programa que usa \ina{getChar} para ler dois
  caracteres, um depois do outro, e imprimir uma cadeia contendo os
  dois caracteres lidos, na ordem em que foram lidos, usando
  \ina{putStr}, usando combinador monádico \ina{(>>=)}. Reescreva
  usando a notação \ina{do}.

\begin{center}
\begin{tabular}{l}
\begin{alg}{.5\textwidth}{white}
  main = getChar >>=
            (\ c1 -> getChar >>=
            \ c2 -> getChar -> putStr [c1,c2]))
\end{alg}
\end{tabular}
\end{center}

Usando a notação \ina{do}:

\begin{center}
\begin{tabular}{l}
\begin{alg}{.5\textwidth}{white}
main = do c1 <- getChar
          c2 <- getChar
          putStr [c1,c2]
\end{alg}
\end{tabular}
\end{center}

\item Considere as duas definições abaixo:

\begin{center}
\begin{tabular}{l}
\begin{alg}{.5\textwidth}{white}
copiaAteLinhaVazia = do lin <- getLine
                        let copia = do
                              if (lin == "")
                                then return ()
                                else do putStrLn lin
                                        lin <- getLine
                                        copia
                        copia
\end{alg}
\end{tabular}
\end{center}

\begin{center}
\begin{tabular}{l}
\begin{alg}{.5\textwidth}{white}
copiaAteLinhaVazia = do 
  l <- getLine
  if null l then return ()
            else do {putStrLn l; copiaAteLinhaVazia}
\end{alg}
\end{tabular}
\end{center}

Uma delas não copia linhas da entrada até que seja lida uma linha
vazia. Qual e porquê?

{\em Solução\/}: A primeira. 

A primeira função lê uma linha e só termina se esta linha for uma
linha vazia; caso contrário, qualquer chamada a
\ina{copiaAteLinhaVazia} não termina: a cláusula \ina{lin <- getLine}
usada após o teste de \ina{(lin == "")}, na notação \ina{do}, não
modifica o valor denotado pela variável \ina{lin} usada nesse
teste. Essa cláusula cria nova variável, de nome \ina{lin}, que tem o
mesmo nome de variável que já existe (em escopo mais externo; a
variável mais externa fica invisível devido à criação da variável com
mesmo nome em escopo mais interno). Podemos reescrever a primeira
função \ina{copiaAteLinhaVazia} sem usar a notação \ina{do} para
tornar isso explícito:

\begin{center}
\begin{tabular}{l}
\begin{alg}{.5\textwidth}{white}
copiaAteLinhaVazia = 
  getLine >>= \ lin -> let copia = if (lin == "")
                                     then return ()
                                     else putStrLn lin >> 
                                          getLine >>= \ lin -> 
                                          copia
                        in copia
\end{alg}
\end{tabular}
\end{center}

\item Escreva uma programa \ina{wc} que se comporte como o programa
  \ina{wc} existente no Linux.  Ou seja, o programa tem o nome de um
  arquivo-texto como argumento e imprime o número de linhas, palavras
  e caracteres existentes no arquivo.

{\em Solução\/}: 

A solução a seguir usa a função \ina{words}, definida em
\ina{Prelude}, para calcular o número de palavras do texto.

\begin{center}
\begin{tabular}{l}
\begin{alg}{.5\textwidth}{white}
wc :: String -> (Int,Int,Int)
wc s = (length s, words s, filter (=='\n') s)

head':: [String] -> String
head' [] = error "Nome de arquivo esperado"
head' s  = head s

main = getArgs >>= print . wc . head' 
\end{alg}
\end{tabular}
\end{center}

A solução a seguir não usa a função \ina{words}; a função \ina{wc'}
percorre a entrada apenas uma vez calculando o número de caracteres,
palavras e linhas (e ilustra como calcular o número de palavras por
meio de um booleano que indica se o caractere corrente é componente de
uma palavra ou não):

\begin{center}
\begin{tabular}{l}
\begin{alg}{.5\textwidth}{white}
import Data.Char (isSpace)

wc :: String -> (Int,Int,Int)
wc = wc' False

wc' _ []                        = (0,0,0)
wc' inWord (c:r) 
  | c_isSpace                   = (numChars+1, numWords', numLins')
  | otherwise                   = (numChars+1, numWords, numLins)
  where 
    c_isSpace                   = isSpace c
    (numChars,numWords,numLins) = wc' (not c_isSpace) r
    numWords'
      | inWord                  = numWords+1 
      | otherwise               = numWords
    numLins'
      | c == '\n'               = numLins+1
      | otherwise               = numLins

head':: [String] -> String
head' [] = error "Nome de arquivo esperado"
head' s  = head s

main = getArgs >>= print . wc . head' 
\end{alg}
\end{tabular}
\end{center}

\item \label{Ex:uso-ST} Escreva uma definição de uma função que receba
  um inteiro positivo \ina{n} e retorne o \ina{n}-ésimo número de
  Fibonacci, usando a mônada \ina{ST} para evitar chamadas recursivas
  de funções. 

\begin{center}
\begin{tabular}{l}
\begin{alg}{.5\textwidth}{white}
fib :: Int -> ST s Integer
fib n = do a <- newSTRef 0
           b <- newSTRef 1
           for_ n (do x <- readSTRef a
                     y <- readSTRef b
                     writeSTRef a y
                     writeSTRef b (x+y)
                 )
           readSTRef a
\end{alg}
\end{tabular}
\end{center}

%A função \ina{($!)} é usada para forçar a avaliação da soma
%(\ina{x+y}); ($!) é definida em \ina{Prelude} como:
%
%\begin{center}
%\begin{tabular}{l}
%\begin{alg}{.5\textwidth}{white}
%($!) :: (a->b) -> a -> b
%f $! x = x `seq` f x
%\end{alg}
%\end{tabular}
%\end{center}

%A função \ina{seq} força a avaliação do argumento (\ina{x}) antes de
%fornecer o resultado da avaliação da aplicação da função (\ina{f}) a
%este argumento. O uso da função \ina{seq} (e portanto \ina{S!}) deve
%ser feito somente para tornar o desempenho mais eficiente (quando
%necessário).

\end{enumerate}

\subsection{Exercícios}
\label{ex-IO}

\begin{enumerate}

\item ... ... 

\end{enumerate}

\section{Mônada \Maybe}
\label{Monada-Maybe}

... ...

\section{Referências na mônada \IO}

O módulo \ina{Data.IORef} contém definições para suporte ao uso de
referências (variáveis no sentido de programação imperativa, i.e.~que
armazenam um valor que pode ser modificado durante a execução de um
programa) em Haskell, na mônada \ina{IO}.

As funções definidas no módulo \ina{Data.IORef} são:

\begin{itemize}
\item \ina{newIORef :: a -> IO (IORef a)}: Recebe um valor e retorna
  uma referência para esse valor.
\item \ina{readIORef :: IORef a -> IO a}: Recebe uma referência e
  retorna o valor apontado pela referência.
\item \ina{writeIORef :: IORef a -> a -> IO()}: Recebe uma
  referência e um valor e retorna uma ação que armazena o valor na
  referência.
\item \ina{modifyIORef :: IORef a -> (a->a) -> IO()}:
  Recebe uma referência e uma função $f$ e retorna uma ação que
  armazena o valor resultante de aplicar $f$ ao valor corrente
  armazenado na referência. Ou seja, modifica o valor $v$ armazenado
  na referência pelo valor dado por $f$ $v$.
\end{itemize}

{\tt \begin{tabbing}
{\em while\/}:: {\em IO Bool\/} -> {\em IO\/}() -> {\em IO\/}()\\
{\em while test action\/} = \\
xx\=\kill
  \> do \=$b$ <- {\em test}\\
  \>    \>if $b$ \=then do \={\em action}\\\
  \>    \>       \>        \>{\em while test action}\\
  \>    \>       \>else return ()
\end{tabbing}}

{\tt readRevWrite = getLine >>= apply rev >>= putLine}

\begin{tabular}[t]{ll} M\^onada-Erro & 
  {\small {\tt \begin{tabular}[t]{l}
    data Err t = Ok t | Error\\
    return a = Ok a\\
    m >>= f = case m of\\
      \ \ \ (Ok a) -> f a \\
      \ \ \ Error -> Error
  \end{tabular} } } 
\end{tabular}   

Para manipular valores sequencialmente, podemos usar combinadores mais
simples que os combinadores monádicos, tipicamente \fmap, da classe
\Functor\ (seção \ref{Funtores}), ou combinadores da classe
\Applicative\ (seção \ref{Applicative}), abordadas a seguir.

%\section{Mônada \ina{ST}}
%\label{Monada-ST}



